% !TEX TS-program = xelatex
\documentclass[11pt]{article}
\usepackage{lindrew}

\title{Sample Project}
\author{Farhan Sadeek}
\date{Last Updated: \today}
\begin{document}
\maketitle
This is a quick document I've put together to help explain how to use my style
package (and for those newer to \LaTeX, a few key facts that are helpful to
know)! If there's anything you think would be useful to add to this document,
feel free to email me at \texttt{sadeek.1@osu.edu} and I'll include that. :)
\textbf{There's an FAQ at the end with the questions I've been asked.}
\section{Commands explanation}
The \texttt{usepackage\{lindrew\}} line above basically means that all of the
commands in the \texttt{lindrew.sty} file are automatically imported into this
document. The first 50 or so lines of that file are some common packages that
are broadly useful for the documents I create, as well as a few customizable
settings (for example, if you instead typed something like
\texttt{usepackage[serif][lindrew]} in line 2, line 16 of my style is not
applied and you get the default font instead.) The middle part of the sty file
is basically a long list of various shortcuts/adjustments that I've found
useful for my own sake -- you can add your own or remove them as you see fit.
Everything after that is for the colored theorem boxes (also customizable).
\newline Many people start their LaTeX documents with a bunch of commands in
the preamble (loading relevant packages, defining shortcuts) which they
copy-paste into every new document they make. (This basically has the same
effect as putting everything in a sty file.) In particular, if you're just
making a bunch of documents for yourself (like I do for lecture notes), it's
fine if you just keep adding more and more shortcuts even if you don't end up
using them all the time; it shouldn't change the compile time too
significantly.

\subsection{Commands explanation}
The main advantage of taking notes in LaTeX notes is that it's fairly easy to
include formulas. If you want to include it seamlessly in your text, use
something like \(\sqrt[3]{2} + \pi \approx 4.4\), while if you want to separate
it out into its own line, use double dollar signs or (my preference for code
readability) brackets as shown below. \\
\[
    \begin{bmatrix}
        1 & 2 \\
        3 & 4
    \end{bmatrix}
    \begin{bmatrix}
        1 \\
        -1
    \end{bmatrix}
    =
    \begin{bmatrix}
        1 & -2 \\
        3 & -4
    \end{bmatrix}
    =
    \begin{bmatrix}
        -1 \\
        -1
    \end{bmatrix}
\]
\\
If you want a formula or string of equalities to span multiple lines, consider using the \texttt{align*} environment and putting ampersands where you want the different lines to line up:

\begin{align*}
    88^2 & = (88 + 12)(88 - 12) + 12^2 \\
         & = 100 \cdot 76 + 12^2       \\
         & = 7600 + 144                \\
         & = 7744                      \\
\end{align*}

(Some people prefer to use \texttt{equation} instead of \texttt{align}, and there's some reasoning \href{https://tex.stackexchange.com/questions/196/eqnarray-vs-align}{here} if you're interested. Personally I don't care about these kinds of details, and by extension some of the stuff in the package might be a little hacky; if it bothers you, feel free to make your own version of my package!)

You can take a look at the \texttt{sty} file for examples of how to create new
shortcuts -- probably by far the most common types are
\texttt{DeclareMathOperator} (so you can type something like
\texttt{\textbackslash ord} and have it show up as \texttt{ord}),
\texttt{newcommand} (to create a new shortcut), and \texttt{renewcommand} (if
the command is already defined). When defining commands, you can either have
them be faster re-definitions of symbols like \(\mathbb{E}\), or you can use
them to simplify common formats you repeatedly need to type like \(\langle \psi
| \phi \rangle\). In the latter case, you can specify a number of arguments
(like the \texttt{bracket} command does) and then specify where they should
show up in the format (that's what \#1, \#2 are for).

\begin{remark*}
    Do keep in mind that whlie shorthand like \texttt{\textbackslash ord} for $\mathbb{R}$ or using \texttt{\texttt{paren}} to get parentheses that automatically wrap around taller expressions like
    \[
        \paren{1 + \frac{1}{1 + \frac{1}{ 1 + \frac{1}{1 + \cdots}}}} = \frac{1 +
            \sqrt{5}}{2}
    \]
    are both very "standard" shortcuts that many people use, some of my shorcuts
    are \textbf{not} standardly named (like \texttt{pars}, which I found useful in
    one physics class and don't think I've ever used since). I should definitely
    mention that nothing I do is particularly "optimized" and you should feel free
    to explore adjustments or alternatives because there are probably are many of
    them that I just haven't needed.
\end{remark*}
\subsection{Using theorem boxes}
(Note that if you don't what the sections to be numbered, you can use \texttt{section*} instead of \texttt{section}. The same is true for all of the boxes below.)
\begin{example}
    Here is an example box.
\end{example}

\begin{theorem}[Pythagorean Theorem]
    Here is an example  box with a particular label.
\end{theorem}

\begin{definition}\label{usefulthing}
    Here is a definition that I might want to label later on.
\end{definition}

Later in the document, you can reference \cref{usefulthing}, and clicking on
the hyperlink will take you back to where it is in the document. The way those
reference are displayed is pretty customizable as well -- feel free to look the
\texttt{hyperref} and \texttt{cleveref} packages if you have storng opinions.

Many people prefer to number their examples vs theorems seperately (so they
don't count up with the same counter) or they prefer to number with respect to
the section or subsection number (so perhaps the above would read Example 1.2.1
instead of Example 1). Those can be controlled by adjusting the
\texttt{numberwithin} or \texttt{sibling} parts of the style package (for
example, not having \texttt{sibling=theorem}) would make them count seperately,
and using \texttt{numberwithin=section} would make everything in Section 1 be
Example 1.1, Theorem 1.2, and so on. You can also load the package as
\texttt{\textbackslash usepackage[formal]\{lindrew\}}, which will automatically
change the numbering.

\section{Other FAQs}
I'll add answers to any questions that people send by email!
\subsection*{How do you get more comfortable what LaTeX commands, formatting, etc?}
If you're unsure how to write a particular symbol, try drawing it on
\href{https://detexify.kirelabs.org/}{Detexify}. For a cool typing game which
gets you familiar with some common symbols, try
\href{https://texnique.xyz/}{TeXnique}. And if you want to make a particular
kind of diagram (e.g. tables, tikz, pictures etc.),
\href{https://overleaf.com}{Overleaf} tends to have good documentation. In
practice, \textbf{I don't think there's any real reason to "learn all the
    commands"}\footnote{Notice that you have to use different quotes for left and right quotes.} ahead of time -- I only figured ever figured things out
when they came up during classes or problem sets I was writing up. It's abita
bit of a tough learning at first, but you'll comfortable pretty quickly if you
keep asking the question ``how can I do this more easily?''.

I'll point out in particular that Overleaf's explanation of
\href{https://bibtex.org}{bibtex} and citation in general is very useful.
Citiing papers, books, etc. is pretty easy with LaTeX, since you can use BibTeX
citiations linked on the pages where they are hosted.
\end{document}