% !TEX TS-program = xelatex
%\documentclass[11pt, draft]{article}
\documentclass[11pt]{article}
\usepackage{lindrew}
\usepackage{xcolor}
\usepackage{amsmath}
\usepackage{amssymb}
\usepackage{tikz}
\usepackage{hyperref}
\usepackage{fontspec}
\title{CSE 2231 - Software II: Software Development and Design}
\author{Lecturer: \textbf{Professor Rob LaTour}\\Lab Hints}
\date{Spring 2025}

\begin{document}
\maketitle

%%%%%%%%%%%%%%%%%%%%%%%%%%%%

\section{January 14, 2025}
The lab for today is just a way to revisit some of the concepts from JUnit
Testing from Software I. Please don't look at the solutions just becuase you
want. We want you to learn these concepts rather than just doing this lab and
calling it a day!

\textbf{Hints for today:}
\begin{itemize}
    \item Think about how your data stucture is storing the data. For example, an empty
          stack is an edge case, and the same with full elements.
    \item \texttt{constructorTest()}: You will create a new instance of the stack being tested.
    \item You need to identify the possible cases, for all the cases.
    \item
\end{itemize}
%\section{January 15, 2025}
%\section{January 17, 2025}
\section{January 21, 2025}
Today we are doing Lab \(5\), called \texttt{Sequence on Stack}. The way you
would approach this will be like following:
\begin{itemize}
    \item Here you have two stacks, and you will use them to create a sequence. So for
          example Stack is FIFO (First In First Out), and you will use the two stacks to
          create a data structure called \texttt{Sequence} where you can add and remove
          elements from both ends.
    \item For \texttt{setLengthOfLeftStack} if the length of the \texttt{leftStack} is
          less than the desired length, you will need to pop elements from the
          \texttt{rightStack} and push them onto the \texttt{leftStack}. If the length of
          the \texttt{leftStack} is greater than the desired length, you will need to pop
          elements from the \texttt{leftStack} and push them onto the
          \texttt{rightStack}.

    \item For \texttt{add} and \texttt{remove} method, you will use the \texttt{setLengthOfLeftStack} method to adjust the length of the left stack add and remove elements accordingly. Here we are using the space between the left and the right stack to indicate our position in the sequence. 
    \item\ No hints for the \texttt{length} method. I hope you all can figure this one out. 
\end{itemize}

\begin{fact}
    If you are using the method `flip' you are doing something wrong! 
\end{fact}
Good luck with labs! Let us know if you have any questions!

\section{January 23, 2025}
I will not be here on this lab but I will update the lab hints for you all.
%\section{January 24, 2025}
%\section{January 27, 2025}
%\section{January 29, 2025}
%\section{January 31, 2025}
%\section{February 3, 2025}
%\section{February 5, 2025}
%\section{February 7, 2025}
%\section{February 10, 2025}
%\section{February 12, 2025}
%\section{February 14, 2025}
%\section{February 17, 2025}
%\section{February 19, 2025}
%\section{February 21, 2025}
%\section{February 24, 2025}
%\section{February 26, 2025}
%\section{February 28, 2025}
%\section{March 3, 2025}
%\section{March 5, 2025}
%\section{March 7, 2025}
%\section{March 17, 2025}
%\section{March 19, 2025}
%\section{March 21, 2025}
%\section{March 24, 2025}
%\section{March 26, 2025}
%\section{March 28, 2025}
%\section{March 31, 2025}
%\section{April 2, 2025}
%\section{April 4, 2025}
%\section{April 7, 2025}
%\section{April 9, 2025}
%\section{April 11, 2025}
%\section{April 14, 2025}
%\section{April 16, 2025}
%\section{April 18, 2025}
%\section{April 21, 2025}
%\section{April 23, 2025}
%\section{April 25, 2025}
%\section{April 28, 2025}

\end{document}
