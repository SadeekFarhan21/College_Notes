% !TEX TS-program = xelatex
\documentclass[11pt]{article}
\usepackage{lindrew}
\usepackage{xcolor}
\usepackage{amsmath}
\usepackage{amssymb}
\usepackage{amsthm}
\usepackage{fontspec}
\title{Math 4547: Real Analysis I}
\author{Lecturer: \textbf{Professor Alex Margolis}\\Notes by: Farhan Sadeek}
\date{Spring 2025}
\begin{document}
\maketitle
\section{January 6, 2025}
Professor Margolis introduced the course and discussed the syllabus. The course
will cover the following topics: Here are some common number systems
\subsection{What is Analysis?}
Analysis is the branch of mathematics that deals with the rigorous study of
limits, functions, derivatives, integrals, and infinite series. It provides the
foundation for calculus and extends its concepts to more abstract settings.

\subsection{Analysis Study Tips}
\begin{itemize}
    \item Attend all lectures and take good notes.
    \item Read the textbook and work through the examples.
    \item Do the homework problems.
    \item Study with others.
    \item Ask questions.
    \item Practice, practice, practice.
\end{itemize}
\begin{theorem}
    Every convergent squence is bounded.
\end{theorem}
\subsection{The Real Numbers}
\subsubsection{What are the reals?}

\begin{itemize}
    \item The \vocab{natural Numbers $\mathbb{N}$} = \{1, 2, 3, \ldots\}
    \item The \vocab{integers $\mathbb{Z}$} = \{0, 1, -1, 2, -2, $\cdots$ \}
    \item The \vocab{rational Numbers $\mathbb{Q}$} = \{$\frac{p}{q}$ | $p,q \in
              \mathbb{Z}$ and $q \neq 0$\}
    \item The \vocab{real Numbers $\mathbb{R}$}
    \item The \vocab{complex Numbers $\mathbb{C}$}: = \{ $a + bi$ | $a,b \in
              \mathbb{R}$\}, where $i^2 = -1$
\end{itemize}

\begin{theorem}
    There is no rational number $x$, such that $x^2 = 2$.
\end{theorem}

\begin{proof}
    We assume for contradiction that such an $x$ exists. Then $x = \frac{p}{q}$ for some $p, q \in \mathbb{Z}$ and $q \neq 0$. We can assume that $p$ and $q$ have no common factors. Then, $\frac{p^2}{q^2} = 2$, which implies \[p^2 = 2q^2\] Thus, $p^2$ is even. As the square of an odd number is odd, it follows $p$ must
    even. Therefore, $p = 2k$ for an integer $k$. We have $2q^2 = p^2 = {(2k)}^2 =
        4k^2$, and so $q^2 = 2k^2$. Thus, $q^2$ is even. Since $p$ and $q$ are both
    even, this contradicts our assumption that $p$ and $q$ have no common factors.
    Therefore, there is no rational number $x$ such that $x^2 = 2$.
\end{proof}

This theorem implies, if we visualize $\mathbb{Q}$ as points lying on a number
line, there is a `hole' where $\sqrt{2}$ is. (There are many more `holes' e.g.
$\pi$, $e$, $\sqrt{3}$, \ldots)

The key property that $\mathbb{R}$ possesses, but $\mathbb{Q}$ doesn't is that
$\mathbb{R}$ has ``no holes'' (formally, $\mathbb{R}$ is complete.)

In this class, we will rigorously deduce all properties of $\mathbb{R}$ from
the axioms of the real numbers.

The axioms are in three groups.

\begin{enumerate}
    \item Field Axioms (addition and multiplication)
    \item Order axioms (needed to describe properties concerning inequalities)
    \item Completeness Axiom
\end{enumerate}

\subsection{Addition axioms}
\begin{enumerate}
    \item For every pair $a, b \in \mathbb{R}$, we can associate a real number $a + b$
          called their \vocab{sum}.
    \item For every real number $a$, there is a real number $-a$ called its
          \vocab{negative} or \vocab{additive inverse}.
    \item There is a special real number $0$ called zero or the additive identity such
          that for all $a, b, c, x, y, z \cdots $ are real numbers unless otherwise
          stated:
          \begin{enumerate}
              \item $a + b = b + a$
              \item $a + (b + c) = (a + b) + c$
              \item $a + 0 = a$
              \item $a + (-a) = 0$
          \end{enumerate}
\end{enumerate}

\section{January 8, 2025}
We will use the axioms to deduce all the properties of $\mathbb{R}$. From these
axioms, we can derive many more properties of the real numbers. \\
\begin{proposition}
    If $x + a$ for all $a \in \mathbb{R}$, then $a = 0$.
\end{proposition}

\begin{proof}
    We know that
    \begin{equation*}
        \begin{aligned}
            x & = x + 0 \quad \text{(A3)}                  \\
              & = 0 + x \quad \text{(by assumption on } x)
        \end{aligned}
    \end{equation*}
\end{proof}

\begin{proposition}{Left cancellation of addition}
    If $a + x = a + y$, then $x = y$.
\end{proposition}

\begin{proof}
    We start with the given equation $a + x = a + y$. By the additive identity property (A3), we have:
    \begin{align*}
        y & = y + 0          & \text{(A3)}    \\
          & = y + (a + (-a)) & \text{(A4)}    \\
          & = (y + a) + (-a) & \text{(A2)}    \\
          & = (a + y) + (-a) & \text{(A1)}    \\
          & = (a + x) + (-a) & \text{(given)} \\
          & = x + (a + (-a)) & \text{(A1)}    \\
          & = x + 0          & \text{(A4)}    \\
          & = x              & \text{(A3)}
    \end{align*}
    Therefore, $x = y$.
\end{proof}

\begin{proposition}
    $-(-a) = a$
\end{proposition}

\begin{proof}
    We need to show that $-(-a) = a$. Consider the following:
    \begin{align*}
        (-a) + (-(-a)) & = 0 \quad \text{(by definition of additive inverse)} \\
        (-a) + a       & = 0 \quad \text{(since } -(-a) = a \text{)}          \\
        a + (-a)       & = 0 \quad \text{(by commutativity of addition)}      \\
        (-a) + (-(-a)) & = a + (-a) \quad \text{(by substitution)}            \\
        (-(-a))        & = a \quad \text{(by left cancellation of addition)}
    \end{align*}
    Therefore, $-(-a) = a$.
\end{proof}
\begin{proposition}
    -(a + b) = (-a) + (-b)
\end{proposition}
\begin{proof}
    We need to show that the additive inverse of $(a + b)$ is equal to the sum of the additive inverses of $a$ and $b$. Consider the following:
    \begin{align*}
        (a + b) + (-(a + b))    & = 0 \quad \text{(by definition of additive inverse)}                  \\
        (a + b) + ((-a) + (-b)) & = a + (b + ((-a) + (-b))) \quad \text{(by associativity of addition)} \\
                                & = a + ((b + (-a)) + (-b)) \quad \text{(by associativity of addition)} \\
                                & = a + ((-a) + (b + (-b))) \quad \text{(by commutativity of addition)} \\
                                & = a + ((-a) + 0) \quad \text{(by definition of additive inverse)}     \\
                                & = a + (-a) \quad \text{(by identity property of addition)}            \\
                                & = 0 \quad \text{(by definition of additive inverse)}
    \end{align*}
    Therefore, $-(a + b) = (-a) + (-b)$.
\end{proof}

\begin{proposition}
    -0 = 0
\end{proposition}
\begin{proof}
    We need to show that the additive inverse of $0$ is $0$. Consider the
    following:
    \begin{align*}
        0 + 0    & = 0 \quad \text{(by the identity property of addition, A3)}  \\
        0 + (-0) & = 0 \quad \text{(by the definition of additive inverse, A4)}
    \end{align*}
    Therefore, we have:
    \begin{align*}
        0 + 0 & = 0 + (-0)
    \end{align*}
    By the left cancellation property of addition, it follows that:
    \begin{align*}
        0 & = -0
    \end{align*}
    Therefore, $-0 = 0$.
\end{proof}

\subsection{Multiplication Axioms}

\begin{definition}
    For all $a, b \in \mathbb{R}$, we can associate a real number $a \times b$ called their \vocab{product}.
\end{definition}

\begin{definition}
    For every $a \in \mathbb{R}$, there is some $a^{-1} \in \mathbb{R}$ called its
    \vocab{multiplicative inverse} or \vocab{reciprocal} such that for all $a \neq 0$, $a \times a^{-1} = 1$.
\end{definition}

\begin{definition}
    There is a number 1 called \vocab{one} or the \vocab{multiplicative identity}
    such that for all $a \in \mathbb{R}$, $a \times 1 = a$.
\end{definition}

\begin{definition}
    For all $a, b, c \in \mathbb{R}$, we have the following properties of multiplication:
    \begin{itemize}
        \item For all $a, b \in \mathbb{R}$, $a \times b = b \times a$.
        \item For all $a, b, c \in \mathbb{R}$, $a \times (b \times c) = (a \times b) \times
                  c$.
        \item For all $a \in \mathbb{R}$, $a \times 0 = 0$.
        \item For all $a, b \in \mathbb{R}$, $a \times (b + c) = a \times b + a \times c$.
    \end{itemize}
\end{definition}

\begin{proposition}
    If $a \times a \times b = a$, and $a \in \RR$ then $b=1$.
\end{proposition}
\begin{proposition}
    If $a \neq 0$ and $a \times b = a \times c$, then $b = c$.
\end{proposition}

\begin{proposition}
    If $a \neq 0$ and $a ^{-1} \neq 0$ and $(a^{-1})^{-1} = a$.
\end{proposition}

\begin{proposition}
    If $a \neq 0$, $b \neq 0$, and $a \times b \neq 0$, then $(a \times b)^{-1} = a^{-1} \times b^{-1}$.
\end{proposition}

\begin{proposition}
    If $a, b, c \in \RR$, then $(a + b) \times c = (a \times b) + (a \times c)$
\end{proposition}

\begin{proof}
    \begin{align*}
        (a + b)\times c & = c \times (a + b)        \\
                        & = c \times a + c \times b \\
                        & = a \times c + b \times c \\
    \end{align*}
\end{proof}

\begin{proposition}
    For all, $a \in \RR, a \times 0 = 0$
\end{proposition}

\begin{proof}
    \begin{align*}
        a \times 0 & = a \times 0 + 0          \\
                   & = a \times (0 + 0)        \\
                   & = a \times 0 + a \times 0 \\
                   & = 0 + 0                   \\
                   & = 0
    \end{align*}
\end{proof}

\begin{proposition}
    If $a \times b = 0$, then either $a = 0$ or $b = 0$ or both.
\end{proposition}
\begin{proposition}
    $a \times (-b) = (a \times b)$. In particular, $a \times (-1) = - -a$
\end{proposition}
\begin{proof}
    \begin{align*}
        a \times (-b) + a \times b & = a \times (b + (-b))          \\
                                   & = a \times 0                   \\
                                   & = 0                            \\
                                   & = a \times b + (-(a \times b))
    \end{align*}
    Hence, additive inverse of $a \times b$ = $-(a \times b)$.
\end{proof}

\begin{proposition}
    $(-1) = -1$ and $(-1) \times (-1) = 1$
\end{proposition}

\begin{proof}
    \begin{align*}
        (-1) \times (-1) & = -(-1) \times 1          \\
                         & = -(-1) \times (1 + 0)    \\
                         & = -(-1) \times (1 + (-1)) \\
                         & = -(-1) \times 0          \\
                         & = 0                       \\
                         & = (-1) + (-1)
    \end{align*}
\end{proof}
%\section{January 10, 2025}
%\section{January 13, 2025}
%\section{January 15, 2025}
%\section{January 17, 2025}
%\section{January 20, 2025}
%\section{January 22, 2025}
%\section{January 24, 2025}
%\section{January 27, 2025}
%\section{January 29, 2025}
%\section{January 31, 2025}
%\section{February 3, 2025}
%\section{February 5, 2025}
%\section{February 7, 2025}
%\section{February 10, 2025}
%\section{February 12, 2025}
%\section{February 14, 2025}
%\section{February 17, 2025}
%\section{February 19, 2025}
%\section{February 21, 2025}
%\section{February 24, 2025}
%\section{February 26, 2025}
%\section{February 28, 2025}
%\section{March 3, 2025}
%\section{March 5, 2025}
%\section{March 7, 2025}
%\section{March 17, 2025}
%\section{March 19, 2025}
%\section{March 21, 2025}
%\section{March 24, 2025}
%\section{March 26, 2025}
%\section{March 28, 2025}
%\section{March 31, 2025}
%\section{April 2, 2025}
%\section{April 4, 2025}
%\section{April 7, 2025}
%\section{April 9, 2025}
%\section{April 11, 2025}
%\section{April 14, 2025}
%\section{April 16, 2025}
%\section{April 18, 2025}
%\section{April 21, 2025}
%\section{April 23, 2025}
%\section{April 25, 2025}
%\section{April 28, 2025}
%\section{April 30, 2025}
%%%%%%%%%%%%%%%%%%%%%%%
\end{document}