\documentclass[11pt]{article}
\usepackage{lindrew}
\usepackage{xcolor}
\usepackage{amsmath}
\usepackage{amssymb}
\usepackage{amsthm}

\title{Math 4547: Real Analysis I}
\author{Lecturer: \textbf{Professor Alex Margolis}\\Notes by: Farhan Sadeek}
\date{Spring 2025}
\begin{document}
\maketitle
\section{January 6, 2025}
Professor Margolis introduced the course and discussed the syllabus. The course
will cover the following topics: Here are some common number systems
\subsection{What is Analysis?}
Analysis is the branch of mathematics that deals with the rigorous study of
limits, functions, derivatives, integrals, and infinite series. It provides the
foundation for calculus and extends its concepts to more abstract settings.

\subsection{Analysis Study Tips}
\begin{itemize}
    \item Attend all lectures and take good notes.
    \item Read the textbook and work through the examples.
    \item Do the homework problems.
    \item Study with others.
    \item Ask questions.
    \item Practice, practice, practice.
\end{itemize}
\begin{theorem}
    Every convergent squence is bounded.
\end{theorem}
\subsection{The Real Numbers}
\subsubsection{What are the reals?}

\begin{itemize}
    \item The \vocab{natural Numbers $\mathbb{N}$} = \{1, 2, 3, \ldots\}
    \item The \vocab{integers $\mathbb{Z}$} = \{0, 1, -1, 2, -2, $\cdots$ \}
    \item The \vocab{rational Numbers $\mathbb{Q}$} = \{$\frac{p}{q}$ | $p,q \in
              \mathbb{Z}$ and $q \neq 0$\}
    \item The \vocab{real Numbers $\mathbb{R}$}
    \item The \vocab{complex Numbers $\mathbb{C}$}: = \{ $a + bi$ | $a,b \in
              \mathbb{R}$\}, where $i^2 = -1$
\end{itemize}

\begin{theorem}
    There is no rational number $x$, such that $x^2 = 2$.
\end{theorem}

\begin{proof}
    We assume for contradiction that such an $x$ exists. Then $x = \frac{p}{q}$ for some $p, q \in \mathbb{Z}$ and $q \neq 0$. We can assume that $p$ and $q$ have no common factors. Then, $\frac{p^2}{q^2} = 2$, which implies \[p^2 = 2q^2\] Thus, $p^2$ is even. As the square of an odd number is odd, it follows $p$ must
    even. Therefore, $p = 2k$ for an integer $k$. We have $2q^2 = p^2 = {(2k)}^2 =
        4k^2$, and so $q^2 = 2k^2$. Thus, $q^2$ is even. Since $p$ and $q$ are both
    even, this contradicts our assumption that $p$ and $q$ have no common factors.
    Therefore, there is no rational number $x$ such that $x^2 = 2$.
\end{proof}

This theorem implies, if we visualize $\mathbb{Q}$ as points lying on a number
line, there is a `hole' where $\sqrt{2}$ is. (There are many more `holes' e.g.
$\pi$, $e$, $\sqrt{3}$, \ldots)

The key property that $\mathbb{R}$ possesses, but $\mathbb{Q}$ doesn't is that
$\mathbb{R}$ has ``no holes'' (formally, $\mathbb{R}$ is complete.)

In this class, we will rigorously deduce all properties of $\mathbb{R}$ from
the axioms of the real numbers.

The axioms are in three groups.

\begin{enumerate}
    \item Field Axioms (addition and multiplication)
    \item Order axioms (needed to describe properties concerning inequalities)
    \item Completeness Axiom
\end{enumerate}

\subsection{Field axioms}
\subsubsection{Addition axioms}
\begin{enumerate}
    \item For every pair $a, b \in \mathbb{R}$, we can associate a real number $a + b$
          called their \vocab{sum}.
    \item For every real number $a$, there is a real number $-a$ called its \vocab{negative}
          or \vocab{additive inverse}.
    \item There is a special real number $0$ called zero or the additive identity such
          that for all $a, b, c$ real numbers:
          \begin{enumerate}
              \item $a + b = b + a$
              \item $a + (b + c) = (a + b) + c$
              \item $a + 0 = a$
              \item $a + (-a) = 0$
          \end{enumerate}
\end{enumerate}

\section{January 8, 2025}

%\section{January 10, 2025}
%\section{January 13, 2025}
%\section{January 15, 2025}
%\section{January 17, 2025}
%\section{January 20, 2025}
%\section{January 22, 2025}
%\section{January 24, 2025}
%\section{January 27, 2025}
%\section{January 29, 2025}
%\section{January 31, 2025}
%\section{February 3, 2025}
%\section{February 5, 2025}
%\section{February 7, 2025}
%\section{February 10, 2025}
%\section{February 12, 2025}
%\section{February 14, 2025}
%\section{February 17, 2025}
%\section{February 19, 2025}
%\section{February 21, 2025}
%\section{February 24, 2025}
%\section{February 26, 2025}
%\section{February 28, 2025}
%\section{March 3, 2025}
%\section{March 5, 2025}
%\section{March 7, 2025}
%\section{March 17, 2025}
%\section{March 19, 2025}
%\section{March 21, 2025}
%\section{March 24, 2025}
%\section{March 26, 2025}
%\section{March 28, 2025}
%\section{March 31, 2025}
%\section{April 2, 2025}
%\section{April 4, 2025}
%\section{April 7, 2025}
%\section{April 9, 2025}
%\section{April 11, 2025}
%\section{April 14, 2025}
%\section{April 16, 2025}
%\section{April 18, 2025}
%\section{April 21, 2025}
%\section{April 23, 2025}
%\section{April 25, 2025}
%\section{April 28, 2025}
%\section{April 30, 2025}
%%%%%%%%%%%%%%%%%%%%%%%
\end{document}