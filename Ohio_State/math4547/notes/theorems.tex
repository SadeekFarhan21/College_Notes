% !TEX TS-program = xelatex

\documentclass[11pt]{article}
\usepackage{lindrew}
\usepackage{xcolor}
\usepackage{amsmath}
\usepackage{amssymb}
\usepackage{amsthm}
\usepackage{atbegshi}
\usepackage{fontspec}
\usepackage{bookmark}
\usepackage{tikz}

\title{Math 4547: Real Analysis I}
\author{Farhan Sadeek}
\begin{document}
\maketitle
\section{The Real Numbers}
\subsection{The Real Numbers}

\begin{theorem}\label{1.1.1}
    There is no rational number \( x \) such that \( x^2 = 2 \).
\end{theorem}
\subsection{Field Axioms}
\subsubsection{Addition Axioms}
\begin{proposition}\label{1.2.1}
    If \( a + x = a \) for all real numbers \( a \), then \( x = 0 \). This means that \( 0 \)
    is the only additive identity.
\end{proposition}

\begin{proposition}\label{1.2.2}
    If \( a + x = a + y \) then \( x = y \). In particular, this means that additive
    inverses are unique.
\end{proposition}
\begin{proposition}\label{1.2.3}
    \( -( -a ) = a \)
\end{proposition}

\begin{proposition}\label{1.2.4}
    \( -( a + b ) = ( -a ) + ( -b ) \)
\end{proposition}

\begin{proposition}\label{1.2.5}
    \( -0 = 0 \)
\end{proposition}
\subsubsection{Multiplication Axioms}
\begin{proposition}\label{1.2.6}
    If \( a \times b = a \) for all real numbers \( a \), then \( b = 1 \).
\end{proposition}

\begin{proposition}\label{1.2.7}
    If \( a \neq 0 \) and \( a \times b = a \times c \), then \( b = c \).
\end{proposition}

\begin{proposition}\label{1.2.8}
    If \( a \neq 0 \), then \( a^{-1} \neq 0 \) and \( \frac{1}{\left(\frac{1}{a}\right)} = a \).
\end{proposition}

\begin{proposition}\label{1.2.9}
    If \( a \neq 0 \), \( b \neq 0 \), and \( a \times b \neq 0 \), then \( \frac{1}{(a \times b)} = \left(\frac{1}{a}\right) \times \left(\frac{1}{b}\right) \).
\end{proposition}

\begin{proposition}\label{1.2.10}
    \( (a + b) \times c = a \times c + b \times c \)
\end{proposition}
\begin{proposition}\label{1.2.11}
    \( a \times 0 = 0 \)
\end{proposition}
\begin{proposition}\label{1.2.12}
    If \( a \times b = 0 \), then either \( a = 0 \) or \( b = 0 \) (or both).
\end{proposition}
\begin{proposition}\label{1.2.13}
    \( a \times (-b) = -(a \times b) \). In particular, note that \( a \times (-1) = -a \).
\end{proposition}

\begin{proposition}\label{1.2.14}
    \( (-1) \times (-1) = 1 \)
\end{proposition}
\subsection{The Order Axioms}
\begin{proposition}\label{1.3.1}
    \( 1 \in P \)
\end{proposition}
\begin{proposition}\label{1.3.2}
    \( a > b \) if and only if \( -a < -b \). In particular, \( x > 0 \) if and only if \( -x < 0 \).
\end{proposition}

\begin{proposition}\label{1.3.3}
    For all real numbers \( x, y, z \):
    \begin{enumerate}
        \item If \( x > 0 \) and \( y > 0 \), then \( x + y > 0 \).
        \item If \( x > 0 \) and \( y > 0 \), then \( x \times y > 0 \).
        \item If \( x > y \) and \( y > z \), then \( x > z \).
    \end{enumerate}
\end{proposition}
\begin{proposition}[Inequalities Shift I]\label{1.3.4}
    Let \( x, y, z \) be reals such that \( x < y \). Then \( x + z < y + z \).
\end{proposition}

\begin{proposition}[Inequalities Shift II]\label{1.3.5} Let \( x, y, z \) be reals such that \( x < y \) and \( 0 < z \). Then \( zx < zy \).
\end{proposition}

\begin{corollary}\label{1.3.6}
    Let \( x, y, z \) be reals such that \( x < y \) and \( z < 0 \). Then \( zx > zy \).
\end{corollary}

\begin{corollary}\label{1.3.7}
    \( a^2 \geq 0 \) for any real \( a \).
\end{corollary}

\begin{proposition}\label{1.3.8}
    If \( x \in P \), then \( 1/x \in P \).
\end{proposition}

\begin{corollary}\label{1.3.9}
    If \( x, y \in P \) and \( x < y \) then \( 1/y < 1/x \).
\end{corollary}

\begin{example}\label{1.3.12}
    \(\max(x, y) = -\min(-x, -y)\)
\end{example}

\begin{theorem}[The Triangle Inequality]\label{1.3.14}
    For any real numbers \( a, b \), we refer to this as the \(\Delta\) inequality.
    \[
        |a + b| \leq |a| + |b|
    \]
    with equality if and only if \( (a \geq 0 \text{ and } b \geq 0) \) or \( (a <
    0 \text{ and } b < 0) \).
\end{theorem}
\begin{proposition}\label{1.3.15}
    \( |ab| = |a||b| \)
\end{proposition}
\begin{theorem}[Bernoulli's Inequality]\label{1.3.16}
     Let \( x \) be a real number with \( x > -1 \) and let \( n \) be a positive integer. Then
    \[
        {(1 + x)}^n \geq 1 + nx
    \]
\end{theorem}
\subsection{Completeness Axiom}
\begin{proposition}\label{1.4.3}
    A maximum (if it exists) of a set \( B \) is unique. Similarly, a minimum is unique.
\end{proposition}
\begin{proposition}\label{1.4.10}
    If \( E \subseteq \mathbb{R} \) has a maximum, then \( \max E = \sup E \).
\end{proposition}
\begin{proposition}[The Approximation Property]\label{1.4.11}
    Let \( E \subseteq \mathbb{R} \) be bounded above and non-empty and let \( \epsilon > 0 \). Then there exists \( x \in E \) such that
    \[
        \sup E - \epsilon < x \leq \sup E
    \]
\end{proposition}
\begin{corollary}\label{1.4.13}
    Let \( E \) be bounded above and non-empty. There is a function \( a : \mathbb{N} \to \mathbb{R} \), such that for all \( n \) we have
    \[
        \sup E - \frac{1}{n} < a(n) \leq \sup E.
    \]
\end{corollary}

\begin{theorem}\label{1.4.14}
    Let \( F \) be a non-empty set which is bounded below. Then the set of lower bounds of \( F \) has a greatest element. This element is known as the greatest lower bound or infimum of \( F \) and is written \( \inf F \).
\end{theorem}
\end{document}