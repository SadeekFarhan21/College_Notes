% !TEX TS-program = xelatex
\documentclass[11pt]{article}
\usepackage{lindrew}

\title{Sample Project}
\author{Farhan Sadeek}
\date{Last Updated: \today}
\begin{document}
\maketitle
This is a quick document I've put together to help explain how to use my style
package (and for those newer to \LaTeX, a few key facts that are helpful to
know)! If there is anything you think would be useful to add to this document,
feel free to email me at \text{sadeek.1@osu.edu}\ and I'll include that :$)$
\textbf{There's an FAQ at the end with the questions I have been asked.}
\section{Commands explanation}
The \texttt{usepackagelindrew} line above basically means that all the
commands in the \texttt{lindrew.sty} file are automatically imported into this
document. The first 50 or so lines of that file are some common packages that
are broadly useful for the documents. I create, as well as a few customizable
settings (for example, if you instead typed something like
\texttt{usepackage[serif][lindrew]} in line 2, line 16 of my style is not
applied and you get the default font instead.) The middle part of the sty file
is basically a long list of various shortcuts / adjustments that I've found
useful for my own sake -- you can add your own or remove them as you see fit.
Everything after that is for the colored theorem boxes (also customizable).
\newline Many people start their LaTeX documents with a bunch of commands in
the preamble (loading relevant packages, defining shortcuts) which they
copy-paste into every new document they make. (This basically has the same
effect as putting everything in a sty file.) In particular, if you're just
making a bunch of documents for yourself (like I do for lecture notes), it's
fine if you just keep addming more and more shortcuts even if you don't end up
using them all the time; it shouldn't change the compile time too
significantly.

\subsection{Commands explanation}
The main advantage of taking notes in LaTeX notes is that it's fairly easy to
include formuals. If you want to include it seamlessly in your text, som
something like \(\sqrt[3]{2} + \pi \approx 4.4\), while if you want to separate
it out into it's use double dollar signs or (my preference for code
readability) brackets as shown below. \\
\[
    \begin{bmatrix}
        1 & 2 \\
        3 & 4
    \end{bmatrix}
    \begin{bmatrix}
        1 \\
        -1
    \end{bmatrix}
    =
    \begin{bmatrix}
        1 & -2  \\
        3 & -44
    \end{bmatrix}
    =
    \begin{bmatrix}
        -1 \\
        -1
    \end{bmatrix}
\]
\\
If you want a formula or string of equalities to span multiple lines, consider using the \texttt{align*} environment and putting ampersands where you want the different lines to line up:

\begin{align*}
    88^2 & = (88 + 12)(88 - 12) + 12^2 \\
         & = 100 \cdot 76 + 12^2       \\
         & = 7600 + 144                \\
         & = 7744                      \\
\end{align*}

(Some people refer to use \texttt{equation} instead of \texttt{align}, and there's some reasoning \href{https://tex.stackexchange.com/questions/196/eqnarray-vs-align}{here} if you're interested. Personally I don't care about thse kids of details, and by extension some of the stuff in the package might be a little hacky; if it bothers you, feel free to make your own version of my package!)

You can take a look at the sty file for examples of how to create new shortcuts - probably by far the most common types are \texttt{DeclareMathOperator} (so you can type something like (\texttt{\ord} and have it show up as ord)). \texttt{newcommand} (to create new shortcut),  


\end{document}