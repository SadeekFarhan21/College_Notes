% !TEX TS-program = xelatex
\documentclass[11pt]{article}
\usepackage{lindrew}

\title{\textbf{18.701: Algebra I}}
\author{\textbf{Lecturer: Professor Mike Artin} \\ Notes by : Farhan Sadeek}
\date{Fall 2018}
\begin{document}
\maketitle
\section{September 5, 2018}
 (Professor Artin wants us to call him Mike). \\
First of all, it's good for us to read through the syllabus. On the back page, there is a diagnostic prblem that is pretty simple; we should have it done by Mondayl. It will count for our grade, but it  required. Apparently the problem sets are hard: we should not expect to finish them quickly.
\begin{fact}
    The two main topics of this semester will be \textbf{group theory} and \textbf{ring theory}.
\end{fact}
When professor Artin was young, he wanted to learn the ``general axioms''. But it's better to use examples and use those to understand the axioms when we're learning mathematics.

\begin{definition}
    The \textcolor{blue}{\textbf{general linear group}}, denoted as $GL_n$, consists of the invertible $n \times n$ matrices with the opreation of matrix multiplication.
\end{definition}

(The definition of a \textcolor{blue}{\textbf{group}} was given on the next day.) To make examples, easier to write down, we'll take $n = 2$. Matrix multiplication looks like the following:
\[
    \begin{bmatrix}
        1 & 2 \\
        3 & 4
    \end{bmatrix}
    \begin{bmatrix}
        5 & 6 \\
        7 & 8
    \end{bmatrix}
    =
    \begin{bmatrix}
        1 \cdot 5 + 2 \cdot 7 & 1 \cdot 6 + 2 \cdot 8 \\
        3 \cdot 5 + 4 \cdot 7 & 3 \cdot 6 + 4 \cdot 8
    \end{bmatrix}
    =
    \begin{bmatrix}
        19 & 22 \\
        43 & 50
    \end{bmatrix}
\]

The definition of matrix multiplication seems kind of complicated, but it turns
out we can come up with a natural explanation. One way to explain this
definition is by looking at column vectors, with matrices ``acting from the
left and side:'' if $V$ is a space of 2-dimensional column vectors, we can
treat our matrix $A$ as a linear operator on $V$, where a vector $v \in V$ gets
sent to $Av$.
\begin{fact}
    Given two matrices $A$ and $B$, it is generally not true that $AB = BA$. (For example, the matrices $\begin{bmatrix}
            1 & 0 \\
            0 & 0
        \end{bmatrix}$
    and $\begin{bmatrix}
            0 & 1 \\
            0 & 0\end{bmatrix}$ do not commute.) However, matrix multiplication is associative (that is $A(BC) = (AB)C$), and we know this because we're just computing three transformations in the same order: $C$, then $B$, then $A$.
\end{fact}

In this class, we'll generally deal with invertible matrices (because they make
our group operations nicer.) By the way, if we don't know this already, the
inverse of a $2 \times 2$ matrix is
\[
    \begin{bmatrix}
        a & b \\
        c & d
    \end{bmatrix} \text{ = } \frac{1}{ad - bc} \begin{bmatrix}
        d  & -b \\
        -c & a
    \end{bmatrix}.
\]

\begin{definition}
    An element is $A$ a group has \textcolor{blue}{\textbf{order}} $n$ if $A^n$ is an identity element.
\end{definition}

\begin{example}
    Consider a matrix $A$ = $
        \begin{bmatrix}
            1  & -1 \\
            -1 & 0
        \end{bmatrix}$
    . Since $A^6 = I$, A is an element of $GL_2$ with oder 6.
\end{example}
Elements can have infinite order as well, but it turns out the space of $2 \times 2 $ matrix is nice:

\begin{theorem}
    If entries of a $2 \times 2$ matrix are rational and the order is finite, it must be 1, 2, 3, 4, or 6.
\end{theorem}

(We'll prove this much later on.) Professor Artin like to use dots instead of zeros in matrices because they looke cleaner, but I will not do this in the notes.

\begin{example}
    The following matrix just cycles the indicies of a vector, so it has $n$ if it is $n$-dimensional.
    \[
        A = \begin{bmatrix}
            0      & 1      & 0      & 0      & \cdots & 0      \\
            0      & 0      & 1      & 0      & \cdots & 0      \\
            0      & 0      & 0      & 1      & \cdots & 0      \\
            \vdots & \vdots & \vdots & \vdots & \ddots & \vdots \\
            0      & 0      & 0      & 0      & \cdots & 1      \\
            1      & 0      & 0      & 0      & \cdots & 0
        \end{bmatrix}_{n \times n}
    \]
\end{example}

There are three kinds of \vocab{elementary matrices}, which basically change
the identity matrix by a tiny bit. (This is idea of \textbf{row reducing}.) We
have the matrices
\[
    \begin{bmatrix}
        1 & a \\
        0 & 1
    \end{bmatrix},
    \begin{bmatrix}
        1 & 0 \\
        a & 1
    \end{bmatrix}
\]
which add $a$ times the scond row and vice versa, the matrices
\[
    \begin{bmatrix}
        c & 0 \\
        0 & 1
    \end{bmatrix},
    \begin{bmatrix}
        1 & 0 \\
        0 & c
    \end{bmatrix}
\]
which multiples one of the two matrices by $c$, and the matrix
\[
    \begin{bmatrix}
        0 & 1 \\
        1 & 0
    \end{bmatrix}
\]
which swaps two rows.

\begin{theorem}
    The elementary matrices genereate $GL_2$. In other words, every $A$ in $GL_2$ is a product of of the above elementary matrices.
\end{theorem}

Let's say we start with an arbitrary matrix $A$ say
\[ M = 
    \begin{bmatrix}
    8 & 5 \\
    4 & 6
    \end{bmatrix}
\]

It's hard to randomly find matrices $E_1, E_2, \cdots $ that multiply $M$. Instead, we should work backwards, and try to write 

\[
E_k \cdots E_2 E_1 M = I
\]

Then we know that $A = E_1^{-1}E_2^{-1}\cdots E_k^{-1}$, since all elementary matrices have elementary matrix inverses. I'm not going to include how to do this here, but we basically work one column at a time and try to get the matrix to the identity.
\end{document}