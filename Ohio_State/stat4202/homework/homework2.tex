% !TEX TS-program = xelatex
\documentclass[11pt]{article}
\usepackage{lindrew}
\usepackage{xcolor}

\usepackage{fontspec}

\title{Stat 4202: Mathematical Statistics II}
\author{\textbf{Homework 2}}
\date{Spring 2025}

\begin{document}

\maketitle
\begin{question}[10.38]
    Show that the estimator of \(\omega = \frac{\sigma_2^2}{\sigma_1^2 + \sigma_2^2}\) is consistent.
\end{question}

\begin{question}
    To show that an estimator can be consistent without being unbiased or even asymptotically unbiased, consider the following estimation procedure: To estimate the mean of a population with the finite variance $\sigma^2$, we first take a random sample of size $n$. Then we randomly draw one of $n$ slips of paper numbered from 1 through $n$, and if the number we draw is 2, 3, \(\ldots\), or $n$, we use as our estimator the mean of the random sample; otherwise, we use the estimate $n^2$. Show that this estimation procedure is
    \begin{itemize}
        \item [(a)] consistent;
        \item [(b)] neither unbiased nor asymptotically unbiased;
    \end{itemize}
\end{question}
\begin{question}
    If $X_1, X_2, \ldots, X_n$ constitute a random sample of size $n$ from an exponential population, show that $\overline{X}$ is a sufficient estimator of the parameter $\theta$.
\end{question}
\begin{question}
    If \(X_1\) and \(X_2\) constitute a random sample of size \(n = 2\) from a Poisson population, show that the mean of the sample is a sufficient estimator of the parameter of \(\lambda\).
\end{question}
\begin{question}
    Given a random sample of size \(n\) from a uniform population with \(\alpha = 0\), find an estimator of \(\beta\) by the method of moments.
\end{question}
\begin{question}
    If \(X_1, X_2, \ldots, X_n\) constitute a random sampmle of size \(n\) from a population given by 
    \[
g(x; \theta, \delta) = \begin{cases}
\frac{1}{\theta} \cdot e^{-\frac{x - \delta}{\theta}} & \text{for } x > \delta \\
0 & \text{elsewhere}
\end{cases}
\]
find estimators for \(\delta\) and \(\delta\) by the method of moments. This distribution is sometimes referred to as the two-parameter exponential distribution, and for \(\theta\) = 1 it is the distribution of Example 3.
\end{question}

\begin{question}
    Given a random sample of size \(n\) from a continous uniform population, use the method of moments to find formulas for the parameters \(\alpha\), and \(\beta\).
\end{question}
\end{document}
