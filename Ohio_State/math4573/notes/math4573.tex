% !TEX TS-program = xelatex
%\documentclass[11pt, draft]{article}
\documentclass[11pt]{article}
\usepackage{lindrew}
\usepackage{xcolor}
\usepackage{amsmath}
\usepackage{amssymb}
\usepackage{tikz}
\usepackage{hyperref}
\usepackage{fontspec}
\title{Math 4573: Number Theory}
\author{Lecturer: \textbf{James Cogdell}\\Notes by: Farhan Sadeek}
\date{Spring 2025}

\begin{document}
\maketitle

%%%%%%%%%%%%%%%%%%%%%%%%%%%%
\section{January 8, 2025}

Dr.\ Cogdell explained the logistics of the class and also took attendance.

\subsection{Conjectures in Number Theory}
\begin{itemize}
    \item A number is divisible by 3 if the sum of its digits is divisible by 3.
    \item \textbf{Fermat's Last Theorem}: There are no three positive integers $a$, $b$, and $c$ that satisfy the equation $a^n + b^n = c^n$ for any integer value of $n$ greater than 2.
    \item There are infinitely many primes.
    \item $\sqrt{2}$ is irrational.
    \item $\pi$ is irrational.
    \item Every number can be written as the sum of four squares (Lagrange's Four Square
          Theorem). For example, $1000 = 10^2 + 30^2 + 0^2 + 0^2$ and $999 = 30^2 + 9^2 +
              3^2 + 3^2$.
    \item The polynomial $n^2 - n + 41$ produces prime numbers for $n = 0, 1, 2, \ldots,
              40$, but not for $n = 41$.
    \item Euler conjectured that no $n^{th}$ power can be written as the sum of two
          $n^{th}$ powers for $n > 2$. This was proven false by the counterexample $144^5
              = 27^5 + 84^5 + 110^5 + 133^5$.
    \item \textbf{Goldbach's Conjecture}: Every even integer greater than 2 can be written as the sum of two primes. For example, $4 = 2 + 2$, $6 = 3 + 3$, $8 = 3 + 5$, $10 = 5 + 5$, $12 = 5 + 7$, $14 = 7 + 7$, $16 = 3 + 13$, $18 = 7 + 11$. This has been verified for numbers up to 100,000 but remains unproven.
\end{itemize}

Number theory is related to \textbf{Abstract Algebra}, but also intersects with
other domains such as \textbf{Combinatorics, Analysis, and Topology}. We will
accept a few fundamental facts about \textbf{Number Theory}.

\begin{fact}
    If \(\mathcal{S}\) is a non-empty set of positive integers, then \(\mathcal{S}\) contains a smallest element. This is known as the Well-Ordering Principle.
\end{fact}

\subsection{Divisibility}
This concept has been known since the time of Euclid.

\begin{definition}
    An integer $b$ is divisible by an integer $a \neq 0$ if there is an integer $x$ such that $b = ax$. We write this as $a \mid b$. If $b$ is not divisible by $a$, we write $a \nmid b$.
\end{definition}

There are two derivative notions:
\begin{itemize}
    \item If $0 < a < b$, then $a$ is called a \textbf{proper divisor} of $b$.
    \item If $a^k \mid\mid b$, it means $a^k \mid b$ and $a^{k + 1} \nmid b$.
\end{itemize}

\begin{theorem}
    Let $a$, $b$, and $c$ be integers. Then the following are true:
    \begin{itemize}
        \item If $a \mid b$, then $a \mid bc$.
        \item If $a \mid b$, then $a \mid b + c$.
        \item If $a \mid b$ and $a \mid c$, then $a \mid b + c$.
        \item If $a \mid b$ and $b \mid a$, then $a = b$ or $a = -b$.
        \item If $a \mid b$ and $a > 0$ and $b > 0$, then $a \leq b$.
        \item If $m \neq 0$ and $a \mid b$, then $am \mid bm$.
        \item If $a \mid b_1, a \mid b_2, \ldots, a \mid b_n$, then $a \mid \sum_{i=1}^{n}
                  b_i x_i$ for any integers $x_i$.
    \end{itemize}
\end{theorem}

\begin{theorem}[The Division Algorithm]
    Given integers \(a\) and \(b\) with \(a > 0\), there exist unique integers \(q\) and \(r\) such that
    \[
        b = qa + r, \quad 0 \leq r < a.
    \]
    If \(a \nmid b\), then \(r\) satisfies the stronger inequality
    \[
        0 < r < a.
    \]
\end{theorem}

\begin{proof}
    Consider the arithmetic progression $\ldots, b - 3a, b - 2a, b - a, b, b + a, b + 2a, b + 3a, \ldots$. In this sequence, select the smallest non-negative member. This defines $r$ and satisfies the inequalities of the theorem. Since $r$ is in the sequence, it can be written as $b - qa$. To prove the uniqueness of $q$ and $r$, suppose there is another pair $q_1$ and $r_1$ that satisfies the same conditions. We first prove that $r = r_1$. If not, assume $r < r_1$, so $0 < r_1 - r < a$. But $r_1 - r = a(q - q_1)$, meaning $a \mid (r_1 - r)$, which contradicts the fact that $0 < r_1 - r < a$. Thus, $r = r_1$ and $q = q_1$.
\end{proof}

\begin{fact}
    If $a \mid b$, then $r$ satisfies the stronger inequality $0 \leq r < a$.
\end{fact}

\begin{fact}
    The Division Algorithm can be stated without the assumption $a > 0$. Given integers $a$ and $b$ with $a \neq 0$, there exist integers $q$ and $r$ such that $b = qa + r$ with $0 \leq |r| < |a|$.
\end{fact}

\begin{definition}[Common Divisor]
    The integer $a$ is a \textbf{common divisor} of $b$ and $c$ if $a \mid b$ and $a \mid c$. Since there is only a finite number of divisors of any non-zero integer, there is only a finite number of common divisors of $b$ and $c$ except in the case $b = c = 0$.
\end{definition}

If at least one of \( b \) and \( c \) is not \( 0 \), the \textbf{greatest
    common divisor} is called the \textbf{gcd} \( \text{gcd}(b, c) \)
(\textit{greatest common divisor of \( b \) and \( c \)}), and is denoted by \(
(b, c) \). Similarly, we have the greatest common divisor \( g \) of the
integers \( b_1, b_2, \ldots, b_n \) (\textit{not all \( 0 \)}) denoted by \(
(b_1, b_2, \ldots, b_n) \).

\begin{theorem}
    If $g$ is the \textbf{gcd} of \( b \) and \( c \), then there exist integers \( x_0 \) and \( y_0 \) such that \[g = bx_0 + cy_0\]
\end{theorem}

\section{January 10, 2025}
Dr.\ Cogdell takes attendance so I will have to be in class every single day.

\begin{definition}[Common Divisor]
    The integer $a$ is a common divisor of $b$ and $c$ if $a \mid b$ and $a \mid c$. Since there is only a finite number of divisors of any nonzero integer, there is only a finite number of common divisors of $b$ and $c$, except in the case $b = c = 0$. If at least one of $b$ and $c$ is not $0$, the greatest among their common divisors is called the greatest common divisor of $b$ and $c$ and is denoted by $(b, c)$. Similarly, we denote the greatest common divisor $g$ of the integers $b_1, b_2, \ldots, b_n$, not all zero, by $(b_1, b_2, \ldots, b_n)$.
\end{definition}

\begin{theorem}
    If $g$ is the greatest common divisor of $b$ and $c$, then there exist integers $x_0$ and $y_0$ such that $g = (b, c) = bx_0 + cy_0$.
\end{theorem}

\begin{fact}
    Another fundamental way to to state this is that the linear combination of \( b \) and \( c \) is with integgral multipliers \( x_0 \) and \( y_0 \). This assertion of holds for any finite collection.
\end{fact}
\begin{proof}
    Consider the following linear combinations \{\(bx + cy\)\} where \(x\) and \(y\) are all integers. Note this also contains \(x = y = 0\). Choose \(bx_0 + cy_0 \) is the least positive integer \(l\) in the set.

    We need to prove that \(l \mid b\) and \(l \mid c\). We will do this via
    indirect proof. If we assume that \(l \nmid b\), we will obtain a
    contradiction. From \(l \nmid b\), there are integers \(q\) and \(r\) such that
    \(b = lq + r\) where \(0 < r < l\). Since \(l\) is the least positive integer
    in the set, we can write \(r = bx_1 + cy_1\) for some integers \(x_1\) and
    \(y_1\). So we have \[r = b - lq = b - q(bx_0 - cy_0) = b(1 - qx_0) + c(-qy_0)\]  and this \(r\) is in the set {\(bx + cy\)}. This contradicts the fact that
    \(l\) is the least positive integer in the set \{\(bx + cy\)\}. Thus, we have
    shown that \(l \mid b\).

    Since \(g\) is the greatest common divisor of \(b\) and \(c\), we may write \(l
    = bx_0 + cy_0 = g(Bx_0 + Cy_0)\). Then, \(g \mid l\) and by parts of theorem
    1.1, we have shown \(g \leq l\). Now, \(g < l\) is impossible since, \(g\)is
    the greatest common divisor, so \(g = l = bx_0 + cy_0\).
\end{proof}

\begin{theorem}
    The greatest common divisor \(g\) of \(b\) and \(c\) can be characterized in the following two ways:
    \begin{itemize}
        \item It is the least positive value of \(bx + cy\) where \(x\) and \(y\) range over
              all integers.
        \item It is the positive common divisor of \(b\) and \(c\) that is divisible by every
              common divisor.
    \end{itemize}
\end{theorem}

\begin{proof}
    Part 1 follows from the proof of Theorem \(1.3\). To prove part 2, we observe that if \(d\) is any common divisor of \(b\) and \(c\), then \(d \mid g\) by part 3 of Theorem \(1.1\). Moreover, there cannot be two distinct integers with property 2, because of Theorem \(1.1\), part 4.
\end{proof}

\begin{remark}
    If an integer \(d\) is expressible in the form \(d = bx + cy\), then \(d\) is not
    necessarily the \(\gcd(b, c)\). However, it does follow from such an equation
    that \((b, c)\) is a divisor of \(d\). In particular, if \(bx + cy = 1\) for some integers
    \(x\) and \(y\), then \((b, c) = 1\).
\end{remark}

\begin{theorem}
    Given any integers $b_1, b_2, \ldots, b_n$ not all zero, with greatest common divisor $g$, there exist integers $x_1, x_2, \ldots, x_n$ such that
    \[
        g = (b_1, b_2, \ldots, b_n) = \sum_{j=1}^{n} b_j x_j.
    \]
    Furthermore, $g$ is the least positive value of the linear form $\sum_{j=1}^{n}
        b_j y_j$ where the $y_j$ range over all integers; also $g$ is the positive
    common divisor of $b_1, b_2, \ldots, b_n$ that is divisible by every common
    divisor.
\end{theorem}
\begin{proof}
    Consider the set \( S = \left\{ \sum_{j=1}^{n} b_j y_j \mid y_j \in \mathbb{Z} \right\} \). Since not all \( b_j \) are zero, there exists a non-zero integer in \( S \). Let \( g \) be the smallest positive integer in \( S \). Then \( g \) can be written as \( g = \sum_{j=1}^{n} b_j x_j \) for some integers \( x_j \).

    We claim that \( g \) is the greatest common divisor of \( b_1, b_2, \ldots,
    b_n \). First, we show that \( g \) is a common divisor of \( b_1, b_2, \ldots,
    b_n \). For each \( b_i \), we have
    \[
        b_i = \sum_{j=1}^{n} b_j \delta_{ij},
    \]
    where \( \delta_{ij} \) is the Kronecker delta. Since \( g \) divides each term
    on the right-hand side, it follows that \( g \mid b_i \) for all \( i \).

    Next, we show that \( g \) is the greatest common divisor. Let \( d \) be any
    common divisor of \( b_1, b_2, \ldots, b_n \). Then \( d \mid \sum_{j=1}^{n}
    b_j x_j \), so \( d \mid g \). Therefore, \( g \) is the greatest common
    divisor of \( b_1, b_2, \ldots, b_n \).

    Finally, we show that \( g \) is the least positive value of the linear form \(
    \sum_{j=1}^{n} b_j y_j \). Suppose there exists a positive integer \( h \) such
    that \( h = \sum_{j=1}^{n} b_j z_j \) and \( h < g \). Then \( h \) is in \( S
    \), which contradicts the minimality of \( g \). Therefore, \( g \) is the
    least positive value of the linear form.

    Thus, we have shown that \( g = (b_1, b_2, \ldots, b_n) = \sum_{j=1}^{n} b_j
    x_j \) and \( g \) is the least positive value of the linear form \(
    \sum_{j=1}^{n} b_j y_j \) where the \( y_j \) range over all integers. Also, \(
    g \) is the positive common divisor of \( b_1, b_2, \ldots, b_n \) that is
    divisible by every common divisor.
\end{proof}

\begin{theorem}
    For any positive integer\(m\) we have \[(ma, mb) = m(a, b)\]
\end{theorem}

\begin{proof}
    By Theorem 1.4 we have
    \begin{align*}
        (ma, mb) & = \text{least positive value of } max + mby           \\
                 & = m \cdot \{\text{least positive value of } ax + by\} \\
                 & = m(a, b).
    \end{align*}
\end{proof}
\begin{theorem}
    If \(d \mid a\) and \(d \mid b\), \(d > 0\), then \[(\frac{a}{b}, \frac{b}{d} ) = \frac{1}{d}(a, b)\]
    If (\(a, b) = g\), then \[(\frac{a}{g}, \frac{b}{g}) = 1\]
\end{theorem}

\begin{proof}
    The second assertion is the special case of the first obtained by using the greatest common divisor \(g\) of \(a\) and \(b\) in the role of \(d\). The first assertion in turn is a direct consequence of Theorem 1.6 obtained by replacing \(m, a, b\) in that theorem by \(d, \frac{a}{d}, \frac{b}{d}\) respectively.
\end{proof}

\begin{theorem}
    If (\(a, m\)) = (\(b, m\)) = 1, then (\(ab, m\)) = 1
\end{theorem}
\begin{proof}
    By Theorem 1.3, there exist integers \(x_0, y_0, x_1, y_1\) such that
    \[1 = ax_0 + my_0 = bx_1 + my_1.\]
    Thus, we may write
    \[
        a x_0 - b x_1 = m (y_1 - y_0).
    \]
    Let \(y_2 = y_1 - y_0\). Then we have
    \[
        a x_0 - b x_1 = m y_2.
    \]
    From the equation \(a x_0 - b x_1 = m y_2\), we note, by part 3 of Theorem 1.1,
    that any common divisor of \(a\) and \(b\) is a divisor of \(m\). Hence, \((a,
    b, m) = 1\).
\end{proof}
%\section{January 13, 2025}
%\section{January 15, 2025}
%\section{January 17, 2025}
%\section{January 20, 2025}
%\section{January 22, 2025}
%\section{January 24, 2025}
%\section{January 27, 2025}
%\section{January 29, 2025}
%\section{January 31, 2025}
%\section{February 3, 2025}
%\section{February 5, 2025}
%\section{February 7, 2025}
%\section{February 10, 2025}
%\section{February 12, 2025}
%\section{February 14, 2025}
%\section{February 17, 2025}
%\section{February 19, 2025}
%\section{February 21, 2025}
%\section{February 24, 2025}
%\section{February 26, 2025}
%\section{February 28, 2025}
%\section{March 3, 2025}
%\section{March 5, 2025}
%\section{March 7, 2025}
%\section{March 17, 2025}
%\section{March 19, 2025}
%\section{March 21, 2025}
%\section{March 24, 2025}
%\section{March 26, 2025}
%\section{March 28, 2025}
%\section{March 31, 2025}
%\section{April 2, 2025}
%\section{April 4, 2025}
%\section{April 7, 2025}
%\section{April 9, 2025}
%\section{April 11, 2025}
%\section{April 14, 2025}
%\section{April 16, 2025}
%\section{April 18, 2025}
%\section{April 21, 2025}
%\section{April 23, 2025}
%\section{April 25, 2025}
%\section{April 28, 2025}

\end{document}
