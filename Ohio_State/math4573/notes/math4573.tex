% !TEX TS-program = xelatex
%\documentclass[11pt, draft]{article}
\documentclass[11pt]{article}
\usepackage{lindrew}
\usepackage{xcolor}
\usepackage{amsmath}
\usepackage{amssymb}
\usepackage{tikz}
\usepackage{hyperref}
\usepackage{fontspec}
\title{Math 4573: Number Theory}
\author{Lecturer: \textbf{James Cogdell}\\Notes by: Farhan Sadeek}
\date{Spring 2025}

\begin{document}
\maketitle

%%%%%%%%%%%%%%%%%%%%%%%%%%%%
\section{January 8, 2025}

Dr. Cogdell explained the logistics of the class and also took attendance.

\subsection{Conjectures in Number Theory}
\begin{itemize}
    \item A number is divisible by 3 if the sum of its digits is divisible by 3.
    \item \textbf{Fermat's Last Theorem}: There are no three positive integers $a$, $b$, and $c$ that satisfy the equation $a^n + b^n = c^n$ for any integer value of $n$ greater than 2.
    \item There are infinitely many primes.
    \item $\sqrt{2}$ is irrational.
    \item $\pi$ is irrational.
    \item Every number can be written as the sum of four squares (Lagrange's Four Square Theorem). For example, $1000 = 10^2 + 30^2 + 0^2 + 0^2$ and $999 = 30^2 + 9^2 + 3^2 + 3^2$.
    \item The polynomial $n^2 - n + 41$ produces prime numbers for $n = 0, 1, 2, \ldots, 40$, but not for $n = 41$.
    \item Euler conjectured that no $n^{th}$ power can be written as the sum of two $n^{th}$ powers for $n > 2$. This was proven false by the counterexample $144^5 = 27^5 + 84^5 + 110^5 + 133^5$.
    \item \textbf{Goldbach's Conjecture}: Every even integer greater than 2 can be written as the sum of two primes. For example, $4 = 2 + 2$, $6 = 3 + 3$, $8 = 3 + 5$, $10 = 5 + 5$, $12 = 5 + 7$, $14 = 7 + 7$, $16 = 3 + 13$, $18 = 7 + 11$. This has been verified for numbers up to 100,000 but remains unproven.
\end{itemize}

Number theory is related to \textbf{Abstract Algebra}, but also intersects with other domains such as \textbf{Combinatorics, Analysis, and Topology}. We will accept a few fundamental facts about \textbf{Number Theory}.

\begin{fact}
    If \(\mathcal{S}\) is a non-empty set of positive integers, then \(\mathcal{S}\) contains a smallest element. This is known as the Well-Ordering Principle.
\end{fact}

\subsection{Divisibility}
This concept has been known since the time of Euclid.

\begin{definition}
    An integer $b$ is divisible by an integer $a \neq 0$ if there is an integer $x$ such that $b = ax$. We write this as $a \mid b$. If $b$ is not divisible by $a$, we write $a \nmid b$.
\end{definition}

There are two derivative notions:
\begin{itemize}
    \item If $0 < a < b$, then $a$ is called a \textbf{proper divisor} of $b$.
    \item If $a^k \mid\mid b$, it means $a^k \mid b$ and $a^{k + 1} \nmid b$.
\end{itemize}

\begin{theorem}
    Let $a$, $b$, and $c$ be integers. Then the following are true:
    \begin{itemize}
        \item If $a \mid b$, then $a \mid bc$.
        \item If $a \mid b$, then $a \mid b + c$.
        \item If $a \mid b$ and $a \mid c$, then $a \mid b + c$.
        \item If $a \mid b$ and $b \mid a$, then $a = b$ or $a = -b$.
        \item If $a \mid b$ and $a > 0$ and $b > 0$, then $a \leq b$.
        \item If $m \neq 0$ and $a \mid b$, then $am \mid bm$.
        \item If $a \mid b_1, a \mid b_2, \ldots, a \mid b_n$, then $a \mid \sum_{i=1}^{n} b_i x_i$ for any integers $x_i$.
    \end{itemize}
\end{theorem}

\begin{theorem}[The Division Algorithm]
    Given integers $a$ and $b$ with $a > 0$, there exist unique integers $q$ and $r$ such that $0 \leq r < a$ and $b = aq + r$.
\end{theorem}

\begin{proof}
    Consider the arithmetic progression $\ldots, b - 3a, b - 2a, b - a, b, b + a, b + 2a, b + 3a, \ldots$. In this sequence, select the smallest non-negative member. This defines $r$ and satisfies the inequalities of the theorem. Since $r$ is in the sequence, it can be written as $b - qa$. To prove the uniqueness of $q$ and $r$, suppose there is another pair $q_1$ and $r_1$ that satisfies the same conditions. We first prove that $r = r_1$. If not, assume $r < r_1$, so $0 < r_1 - r < a$. But $r_1 - r = a(q - q_1)$, meaning $a \mid (r_1 - r)$, which contradicts the fact that $0 < r_1 - r < a$. Thus, $r = r_1$ and $q = q_1$.
\end{proof}

\begin{fact}
    If $a \mid b$, then $r$ satisfies the stronger inequality $0 \leq r < a$.
\end{fact}

\begin{fact}
    The Division Algorithm can be stated without the assumption $a > 0$. Given integers $a$ and $b$ with $a \neq 0$, there exist integers $q$ and $r$ such that $b = qa + r$ with $0 \leq |r| < |a|$.
\end{fact}


%\section{January 10, 2025}
%\section{January 13, 2025}
%\section{January 15, 2025}
%\section{January 17, 2025}
%\section{January 20, 2025}
%\section{January 22, 2025}
%\section{January 24, 2025}
%\section{January 27, 2025}
%\section{January 29, 2025}
%\section{January 31, 2025}
%\section{February 3, 2025}
%\section{February 5, 2025}
%\section{February 7, 2025}
%\section{February 10, 2025}
%\section{February 12, 2025}
%\section{February 14, 2025}
%\section{February 17, 2025}
%\section{February 19, 2025}
%\section{February 21, 2025}
%\section{February 24, 2025}
%\section{February 26, 2025}
%\section{February 28, 2025}
%\section{March 3, 2025}
%\section{March 5, 2025}
%\section{March 7, 2025}
%\section{March 17, 2025}
%\section{March 19, 2025}
%\section{March 21, 2025}
%\section{March 24, 2025}
%\section{March 26, 2025}
%\section{March 28, 2025}
%\section{March 31, 2025}
%\section{April 2, 2025}
%\section{April 4, 2025}
%\section{April 7, 2025}
%\section{April 9, 2025}
%\section{April 11, 2025}
%\section{April 14, 2025}
%\section{April 16, 2025}
%\section{April 18, 2025}
%\section{April 21, 2025}
%\section{April 23, 2025}
%\section{April 25, 2025}
%\section{April 28, 2025}

\end{document}
