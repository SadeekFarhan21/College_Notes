% !TEX TS-program = xelatex
%\documentclass[11pt, draft]{article}
\documentclass[11pt]{article}
\usepackage{lindrew}
\usepackage{xcolor}
\usepackage{amsmath}
\usepackage{amssymb}
\usepackage{tikz}
\usepackage{hyperref}
\usepackage{fontspec}
\title{Math 4573: Number Theory}
\author{Lecturer: \textbf{Profesor James Cogdell}\\Notes by: Farhan Sadeek}
\date{Spring 2025}

\begin{document}
\maketitle

%%%%%%%%%%%%%%%%%%%%%%%%%%%%
\section{January 8, 2025}

Dr.\ Cogdell explained the logistics of the class and also took attendance.
This class will be no exams and graded based on only homeworks.

\subsection{Conjectures in Number Theory}
\begin{itemize}
    \item A number is divisible by 3 if the sum of its digits is divisible by 3.
    \item \textbf{Fermat's Last Theorem}: There are no three positive integers $a$, $b$, and $c$ that satisfy the equation $a^n + b^n = c^n$ for any integer value of $n$ greater than 2.
    \item There are infinitely many primes.
    \item $\sqrt{2}$ is irrational.
    \item $\pi$ is irrational.
    \item Every number can be written as the sum of four squares (Lagrange's Four Square
          Theorem). For example, $1000 = 10^2 + 30^2 + 0^2 + 0^2$ and $999 = 30^2 + 9^2 +
              3^2 + 3^2$.
    \item The polynomial $n^2 - n + 41$ produces prime numbers for $n = 0, 1, 2, \ldots,
              40$, but not for $n = 41$.
    \item Euler conjectured that no $n^{th}$ power can be written as the sum of two
          $n^{th}$ powers for $n > 2$. This was proven false by the counterexample $144^5
              = 27^5 + 84^5 + 110^5 + 133^5$.
    \item \textbf{Goldbach's Conjecture}: Every even integer greater than 2 can be written as the sum of two primes. For example, $4 = 2 + 2$, $6 = 3 + 3$, $8 = 3 + 5$, $10 = 5 + 5$, $12 = 5 + 7$, $14 = 7 + 7$, $16 = 3 + 13$, $18 = 7 + 11$. This has been verified for numbers up to 100,000 but remains unproven.
\end{itemize}

Number theory is related to \textbf{Abstract Algebra}, but also intersects with
other domains such as \textbf{Combinatorics, Analysis, and Topology}. We will
accept a few fundamental facts about \textbf{Number Theory}.

\begin{fact}
    If \(\mathcal{S}\) is a non-empty set of positive integers, then \(\mathcal{S}\) contains a smallest element. This is known as the Well-Ordering Principle.
\end{fact}

\subsection{Divisibility}
This concept has been known since the time of Euclid.

\begin{definition}
    An integer $b$ is divisible by an integer $a \neq 0$ if there is an integer $x$ such that $b = ax$. We write this as $a \mid b$. If $b$ is not divisible by $a$, we write $a \nmid b$.
\end{definition}

There are two derivative notions:
\begin{itemize}
    \item If $0 < a < b$, then $a$ is called a \textbf{proper divisor} of $b$.
    \item If $a^k \mid\mid b$, it means $a^k \mid b$ and $a^{k + 1} \nmid b$.
\end{itemize}

\begin{theorem}
    Let $a$, $b$, and $c$ be integers. Then the following are true:
    \begin{itemize}
        \item If $a \mid b$, then $a \mid bc$.
        \item If $a \mid b$, then $a \mid b + c$.
        \item If $a \mid b$ and $a \mid c$, then $a \mid b + c$.
        \item If $a \mid b$ and $b \mid a$, then $a = b$ or $a = -b$.
        \item If $a \mid b$ and $a > 0$ and $b > 0$, then $a \leq b$.
        \item If $m \neq 0$ and $a \mid b$, then $am \mid bm$.
        \item If $a \mid b_1, a \mid b_2, \ldots, a \mid b_n$, then $a \mid \sum_{i=1}^{n}
                  b_i x_i$ for any integers $x_i$.
    \end{itemize}
\end{theorem}

\begin{theorem}[The Division Algorithm]
    Given integers \(a\) and \(b\) with \(a > 0\), there exist unique integers \(q\) and \(r\) such that
    \[
        b = qa + r, \quad 0 \leq r < a.
    \]
    If \(a \nmid b\), then \(r\) satisfies the stronger inequality
    \[
        0 < r < a.
    \]
\end{theorem}

\begin{proof}
    Consider the arithmetic progression $\ldots, b - 3a, b - 2a, b - a, b, b + a, b + 2a, b + 3a, \ldots$. In this sequence, select the smallest non-negative member. This defines $r$ and satisfies the inequalities of the theorem. Since $r$ is in the sequence, it can be written as $b - qa$. To prove the uniqueness of $q$ and $r$, suppose there is another pair $q_1$ and $r_1$ that satisfies the same conditions. We first prove that $r = r_1$. If not, assume $r < r_1$, so $0 < r_1 - r < a$. But $r_1 - r = a(q - q_1)$, meaning $a \mid (r_1 - r)$, which contradicts the fact that $0 < r_1 - r < a$. Thus, $r = r_1$ and $q = q_1$.
\end{proof}

\begin{fact}
    If $a \mid b$, then $r$ satisfies the stronger inequality $0 \leq r < a$.
\end{fact}

\begin{fact}
    The Division Algorithm can be stated without the assumption $a > 0$. Given integers $a$ and $b$ with $a \neq 0$, there exist integers $q$ and $r$ such that $b = qa + r$ with $0 \leq |r| < |a|$.
\end{fact}

\begin{definition}[Common Divisor]
    The integer $a$ is a \textbf{common divisor} of $b$ and $c$ if $a \mid b$ and $a \mid c$. Since there is only a finite number of divisors of any non-zero integer, there is only a finite number of common divisors of $b$ and $c$ except in the case $b = c = 0$.
\end{definition}

If at least one of \( b \) and \( c \) is not \( 0 \), the \textbf{greatest
    common divisor} is called the \textbf{gcd} \( \text{gcd}(b, c) \)
(\textit{greatest common divisor of \( b \) and \( c \)}), and is denoted by \(
(b, c) \). Similarly, we have the greatest common divisor \( g \) of the
integers \( b_1, b_2, \ldots, b_n \) (\textit{not all \( 0 \)}) denoted by \(
(b_1, b_2, \ldots, b_n) \).

\begin{theorem}
    If $g$ is the \textbf{gcd} of \( b \) and \( c \), then there exist integers \( x_0 \) and \( y_0 \) such that \[g = bx_0 + cy_0\]
\end{theorem}

\section{January 10, 2025}
Dr.\ Cogdell takes attendance so I will have to be in class every single day.

\begin{definition}[Common Divisor]
    The integer $a$ is a common divisor of $b$ and $c$ if $a \mid b$ and $a \mid c$. Since there is only a finite number of divisors of any nonzero integer, there is only a finite number of common divisors of $b$ and $c$, except in the case $b = c = 0$. If at least one of $b$ and $c$ is not $0$, the greatest among their common divisors is called the greatest common divisor of $b$ and $c$ and is denoted by $(b, c)$. Similarly, we denote the greatest common divisor $g$ of the integers $b_1, b_2, \ldots, b_n$, not all zero, by $(b_1, b_2, \ldots, b_n)$.
\end{definition}

\begin{theorem}
    If $g$ is the greatest common divisor of $b$ and $c$, then there exist integers $x_0$ and $y_0$ such that $g = (b, c) = bx_0 + cy_0$.
\end{theorem}

\begin{fact}
    Another fundamental way to to state this is that the linear combination of \( b \) and \( c \) is with integgral multipliers \( x_0 \) and \( y_0 \). This assertion of holds for any finite collection.
\end{fact}
\begin{proof}
    Consider the following linear combinations \{\(bx + cy\)\} where \(x\) and \(y\) are all integers. Note this also contains \(x = y = 0\). Choose \(bx_0 + cy_0 \) is the least positive integer \(l\) in the set.

    We need to prove that \(l \mid b\) and \(l \mid c\). We will do this via
    indirect proof. If we assume that \(l \nmid b\), we will obtain a
    contradiction. From \(l \nmid b\), there are integers \(q\) and \(r\) such that
    \(b = lq + r\) where \(0 < r < l\). Since \(l\) is the least positive integer
    in the set, we can write \(r = bx_1 + cy_1\) for some integers \(x_1\) and
    \(y_1\). So we have \[r = b - lq = b - q(bx_0 - cy_0) = b(1 - qx_0) + c(-qy_0)\]  and this \(r\) is in the set {\(bx + cy\)}. This contradicts the fact that
    \(l\) is the least positive integer in the set \{\(bx + cy\)\}. Thus, we have
    shown that \(l \mid b\).

    Since \(g\) is the greatest common divisor of \(b\) and \(c\), we may write \(l
    = bx_0 + cy_0 = g(Bx_0 + Cy_0)\). Then, \(g \mid l\) and we have shown \(g \leq
    l\). Now, \(g < l\) is impossible since, \(g\)is the greatest common divisor,
    so \(g = l = bx_0 + cy_0\).
\end{proof}

\begin{theorem}
    The greatest common divisor \(g\) of \(b\) and \(c\) can be characterized in the following two ways:
    \begin{itemize}
        \item It is the least positive value of \(bx + cy\) where \(x\) and \(y\) range over
              all integers.
        \item It is the positive common divisor of \(b\) and \(c\) that is divisible by every
              common divisor.
    \end{itemize}
\end{theorem}

\begin{proof}
    Part 1 follows from the proof of Theorem \(1.3\). To prove part 2, we observe that if \(d\) is any common divisor of \(b\) and \(c\), then \(d \mid g\) by part 3 of Theorem \(1.1\). Moreover, there cannot be two distinct integers with property 2, because of Theorem \(1.1\), part 4.
\end{proof}

\begin{remark}
    If an integer \(d\) is expressible in the form \(d = bx + cy\), then \(d\) is not
    necessarily the \(\gcd(b, c)\). However, it does follow from such an equation
    that \((b, c)\) is a divisor of \(d\). In particular, if \(bx + cy = 1\) for some integers
    \(x\) and \(y\), then \((b, c) = 1\).
\end{remark}

\begin{theorem}
    Given any integers $b_1, b_2, \ldots, b_n$ not all zero, with greatest common divisor $g$, there exist integers $x_1, x_2, \ldots, x_n$ such that
    \[
        g = (b_1, b_2, \ldots, b_n) = \sum_{j=1}^{n} b_j x_j.
    \]
    Furthermore, $g$ is the least positive value of the linear form $\sum_{j=1}^{n}
        b_j y_j$ where the $y_j$ range over all integers; also $g$ is the positive
    common divisor of $b_1, b_2, \ldots, b_n$ that is divisible by every common
    divisor.
\end{theorem}
\begin{proof}
    Consider the set \( S = \left\{ \sum_{j=1}^{n} b_j y_j \mid y_j \in \mathbb{Z} \right\} \). Since not all \( b_j \) are zero, there exists a non-zero integer in \( S \). Let \( g \) be the smallest positive integer in \( S \). Then \( g \) can be written as \( g = \sum_{j=1}^{n} b_j x_j \) for some integers \( x_j \).

    We claim that \( g \) is the greatest common divisor of \( b_1, b_2, \ldots,
    b_n \). First, we show that \( g \) is a common divisor of \( b_1, b_2, \ldots,
    b_n \). For each \( b_i \), we have
    \[
        b_i = \sum_{j=1}^{n} b_j \delta_{ij},
    \]
    where \( \delta_{ij} \) is the Kronecker delta. Since \( g \) divides each term
    on the right-hand side, it follows that \( g \mid b_i \) for all \( i \).

    Next, we show that \( g \) is the greatest common divisor. Let \( d \) be any
    common divisor of \( b_1, b_2, \ldots, b_n \). Then \( d \mid \sum_{j=1}^{n}
    b_j x_j \), so \( d \mid g \). Therefore, \( g \) is the greatest common
    divisor of \( b_1, b_2, \ldots, b_n \).

    Finally, we show that \( g \) is the least positive value of the linear form \(
    \sum_{j=1}^{n} b_j y_j \). Suppose there exists a positive integer \( h \) such
    that \( h = \sum_{j=1}^{n} b_j z_j \) and \( h < g \). Then \( h \) is in \( S
    \), which contradicts the minimality of \( g \). Therefore, \( g \) is the
    least positive value of the linear form.

    Thus, we have shown that \( g = (b_1, b_2, \ldots, b_n) = \sum_{j=1}^{n} b_j
    x_j \) and \( g \) is the least positive value of the linear form \(
    \sum_{j=1}^{n} b_j y_j \) where the \( y_j \) range over all integers. Also, \(
    g \) is the positive common divisor of \( b_1, b_2, \ldots, b_n \) that is
    divisible by every common divisor.
\end{proof}

\begin{theorem}
    For any positive integer\(m\) we have \[(ma, mb) = m(a, b)\]
\end{theorem}

\begin{proof}
    By Theorem 1.4 we have
    \begin{align*}
        (ma, mb) & = \text{least positive value of } max + mby           \\
                 & = m \cdot \{\text{least positive value of } ax + by\} \\
                 & = m(a, b).
    \end{align*}
\end{proof}
\begin{theorem}
    If \(d \mid a\) and \(d \mid b\), \(d > 0\), then \[(\frac{a}{b}, \frac{b}{d} ) = \frac{1}{d}(a, b)\]
    If (\(a, b) = g\), then \[(\frac{a}{g}, \frac{b}{g}) = 1\]
\end{theorem}

\begin{proof}
    The second assertion is the special case of the first obtained by using the greatest common divisor \(g\) of \(a\) and \(b\) in the role of \(d\). The first assertion in turn is a direct consequence of Theorem 1.6 obtained by replacing \(m, a, b\) in that theorem by \(d, \frac{a}{d}, \frac{b}{d}\) respectively.
\end{proof}

\begin{theorem}
    If (\(a, m\)) = (\(b, m\)) = 1, then (\(ab, m\)) = 1
\end{theorem}
\begin{proof}
    By Theorem 1.3, there exist integers \(x_0, y_0, x_1, y_1\) such that
    \[1 = ax_0 + my_0 = bx_1 + my_1.\]
    Thus, we may write
    \[
        a x_0 - b x_1 = m (y_1 - y_0).
    \]
    Let \(y_2 = y_1 - y_0\). Then we have
    \[
        a x_0 - b x_1 = m y_2.
    \]
    From the equation \(a x_0 - b x_1 = m y_2\), we note, by part 3 of Theorem 1.1,
    that any common divisor of \(a\) and \(b\) is a divisor of \(m\). Hence, \((a,
    b, m) = 1\).
\end{proof}
\section{January 13, 2025}
\subsection{Euclidean Algorithm}
Given two integers \(b\) and \(c\), now we can generate the greatest common
divisor. There is no algorithm to this problem, but there is an algorithm.

\begin{question}
    Given a set of integers (\(bx + cy\)) how to find the greatest common divisor?
\end{question}
Consider the case \(b = 963\) and \(c = 657\). If we divide \(c\) into \(b\), we get the quotient \(q = 1\) and the remainder \(r = 306\). We can write this as \(b = qc + r\) or \(r = b - cq\). In particular, \(306 = 963 - 1 \cdot 657\). Now \((b, c) = (b - cq, c)\) by replacing \(a\) and \(x\) by \(c\) and \(-q\) in Theorem 1.9, so we see that
\[
    (963, 657) = (963 - 1 \cdot 657, 657) = (306, 657).
\]
The integer 963 has been replaced by the smaller integer 306, and this suggests
that the procedure be repeated. So we divide 306 into 657 to get a quotient 2
and a remainder 45, and
\[
    (306, 657) = (306, 657 - 2 \cdot 306) = (306, 45).
\]
Next, 45 is divided into 306 with quotient 6 and remainder 36, then 36 is
divided into 45 with quotient 1 and remainder 9. We conclude that
\[
    (963, 657) = (306, 657) = (306, 45) = (45, 36) = (36, 9).
\]
Thus \((963, 657) = 9\), and we can express 9 as a linear combination of 963
and 657 by sequentially writing each remainder as a linear combination of the
two original numbers:
\[
    \begin{aligned}
        306 & = 963 - 657,                                                                                          \\
        45  & = 657 - 2 \cdot 306 = 657 - 2 \cdot (963 - 657) = 3 \cdot 657 - 2 \cdot 963,                          \\
        36  & = 306 - 6 \cdot 45 = (963 - 657) - 6 \cdot (3 \cdot 657 - 2 \cdot 963) = 13 \cdot 963 - 19 \cdot 657, \\
        9   & = 45 - 36 = 3 \cdot 657 - 2 \cdot 963 - (13 \cdot 963 - 19 \cdot 657) = 22 \cdot 657 - 15 \cdot 963.
    \end{aligned}
\]
In terms of Theorem 1.3, where \(g = (b, c) = bx_0 + cy_0\), beginning with \(b
= 963\) and \(c = 657\) we have used a procedure called the Euclidean algorithm
to find \(g = 9\), \(x_0 = -15\), \(y_0 = 22\). Of course, these values for
\(x_0\) and \(y_0\) are not unique: \(-15 + 657k\) and \(22 - 963k\) will do
where \(k\) is any integer.

To find the greatest common divisor \((b, c)\) of any two integers \(b\) and
\(c\), we now generalize what is done in the special case above. The process
will also give integers \(x_0\) and \(y_0\) satisfying the equation \(bx_0 +
cy_0 = (b, c)\). The case \(c = 0\) is special: \((b, 0) = |b|\). For \(c \neq
0\), we observe that \((b, c) = (b, -c)\) by Theorem 1.9, and hence, we may
presume that \(c\) is positive.

\begin{theorem}[The Euclidean Algorithm]
    Given integers \( b \) and \( c > 0 \), we make a repeated application of the division algorithm, Theorem 1.2, to obtain a series of equations:
    \[
        \begin{aligned}
            b       & = cq_1 + r_1, \quad 0 < r_1 < c,                                         \\
            c       & = r_1q_2 + r_2, \quad 0 < r_2 < r_1,                                     \\
            r_1     & = r_2q_3 + r_3, \quad 0 < r_3 < r_2,                                     \\
                    & \ \, \vdots \quad \quad \quad \quad \quad \quad \quad \quad \quad \vdots \\
            r_{j-2} & = r_{j-1}q_j + r_j, \quad 0 < r_j < r_{j-1},                             \\
            r_{j-1} & = r_jq_{j+1}.
        \end{aligned}
    \]

    The greatest common divisor \((b, c)\) of \( b \) and \( c \) is \( r_j \), the
    last nonzero remainder in the division process. Values of \( x_0 \) and \( y_0
    \) in \((b, c) = bx_0 + cy_0 \) can be obtained by writing each \( r_i \) as a
    linear combination of \( b \) and \( c \).
\end{theorem}

\begin{proof}
    The chain of equations is obtained by dividing \(c\) into \(b\), \(r_1\) into \(c\), \(r_2\) into \(r_1\), and so on, until \(r_j\) into \(r_{j-1}\). The process stops when the division is exact, that is, when the remainder is zero. Thus, in our application of Theorem 1.2, we have written the inequalities for the remainder without an equality sign. For example, \(0 < r_1 < c\) instead of \(0 \leq r_1 < c\), because if \(r_1\) were equal to zero, the chain would stop at the first equation \(b = cq_1\), in which case the greatest common divisor of \(b\) and \(c\) would be \(c\).

    We now prove that \(r_j\) is the greatest common divisor \(g\) of \(b\) and
    \(c\). By Theorem 1.9, we observe that
    \[
        (b, c) = (c, r_1) = (r_1, r_2) = \cdots = (r_{j-1}, r_j) = (r_j, 0) = r_j.
    \]

    To see that \(r_j\) is a linear combination of \(b\) and \(c\), we argue by
    induction that each \(r_i\) is a linear combination of \(b\) and \(c\).
    Clearly, \(r_1\) is such a linear combination, and likewise \(r_2\). In
    general, \(r_i\) is a linear combination of \(r_{i-1}\) and \(r_{i-2}\). By the
    inductive hypothesis, we may suppose that these latter two numbers are linear
    combinations of \(b\) and \(c\), and it follows that \(r_i\) is also a linear
    combination of \(b\) and \(c\).
\end{proof}

\section{January 15, 2025}
\begin{example}
    We will find the g.c.d of 42823 and 6409.
\end{example}
\begin{solution}

    We apply the Euclidean algorithm to divide \(c\) into \(b\), where \(b =
    42823\) and \(c = 6409\). We obtain a quotient \(q_1 = 6\) and a remainder
    \(r_1 = 4369\). Continuing, if we divide 4369 into 6409, we get a quotient
    \(q_2 = 1\) and a remainder \(r_2 = 2040\). Dividing 2040 into 4369 gives \(q_3
    = 2\) and \(r_3 = 289\). Dividing 289 into 2040 gives \(q_4 = 7\) and \(r_4 =
    17\). Since 17 is an exact divisor of 289, the solution is that the g.c.d. is
    17.

    We can write this in tabular form:
    \[
        \begin{aligned}
            42823 & = 6 \cdot 6409 + 4369, \\
            6409  & = 1 \cdot 4369 + 2040, \\
            4369  & = 2 \cdot 2040 + 289,  \\
            2040  & = 7 \cdot 289 + 17,    \\
            289   & = 17 \cdot 17.
        \end{aligned}
    \]
    Thus, \((42823, 6409) = (6409, 4369) = (4369, 2040) = (2040, 289) = (289, 17) =
    17\).
\end{solution}

\begin{example}
    Find integers \(x\) and \(y\) such that \(42823x + 6409y = 17\).
\end{example}
\begin{solution}

    We find integers \(x\) and \(y\) such that \(42823x + 6409y = 17\).

    Here it is natural to consider \(i = 1, 2, \ldots\), but to initiate the
    process we also consider \(i = 0\) and \(i = -1\). We put \(r_{-1} = 42823\),
    and write
    \[42823 \cdot 1 + 6409 \cdot 0 = 42823.\]
    Similarly, we put \(r_0 = 6409\), and write
    \[42823 \cdot 0 + 6409 \cdot 1 = 6409.\]
    We multiply the second of these equations by \(q_1 = 6\), and subtract the
    result from the first equation, to obtain
    \[42823 \cdot 1 + 6409 \cdot (-6) = 4369.\]
    We multiply this equation by \(q_2 = 1\), and subtract it from the preceding
    equation to find that
    \[42823 \cdot (-1) + 6409 \cdot 7 = 2040.\]
    We multiply this by \(q_3 = 2\), and subtract the result from the preceding
    equation to find that
    \[42823 \cdot 3 + 6409 \cdot (-20) = 289.\]
    Next we multiply this by \(q_4 = 7\), and subtract the result from the
    preceding equation to find that
    \[42823 \cdot (-22) + 6409 \cdot 147 = 17.\]
    On dividing 17 into 289, we find that \(q_5 = 17\) and that \(289 = 17 \cdot
    17\). Thus \(r_4\) is the last positive remainder, so that \(g = 17\), and we
    may take \(x = -22\), \(y = 147\). These values of \(x\) and \(y\) are not the
    only ones possible. In Section 5.1, an analysis of all solutions of a linear
    equation is given.
\end{solution}

\begin{remark}
    Section 5.1 on Analysis
\end{remark}

\begin{definition}
    The integers \(a_1, a_2, \ldots , a_n\) all different from zero, have a common \(b\) if \(a_i \mid b\) for \(i = 1, 2, \ldots, n\). The least positive multiple is is called \vocab{least common multiple} and it's denoted \([a_1, a_2, \ldots, a_n]\)
\end{definition}

\begin{theorem}
    If \(b\) is any common multiple of \(a_1, a_2, \ldots, a_n\), then \([a_1, a_2, \ldots, a_n] \mid b\). This is the same as saying that if \(h\) denotes \([a_1, a_2, \ldots, a_n]\), then \(0, \pm h, \pm 2h, \pm 3, \ldots\) comprise all the common multiples of \(a_1, a_2, \ldots, a_n\).
\end{theorem}
\begin{proof}
    Let \(m\) be any common multiple and divide \(m\) by \(h\). By Division Algorithm, there is a quotient \(q\) and a remainder \(r\) such that \(m = qh + r\), where \(0 \leqslant r < h\). We must prove that \(r = 0\). If \(r \neq -\), we argue as follows. For each \(i = 1, 2, \ldots, n\), we know that \(a_i \mid h\) and \(a_i \mid m\), so that \(a_i \mid r\). Thus \(r\) is a positive common multiple of \(a_1, a_2, \ldots, a_n\) contrary to the fact that \(h\) is the least of all common positive multiple.
\end{proof}

\begin{theorem}
    If \(m > 0\) \([ma, mb] = m[a, b]\). Also, \([a,b] \cdot (a, b) = |ab|\)
\end{theorem}
\begin{proof}
    Let \( H = [ma, mb] \) and \( h = [a, b] \). Then \( mh \) is a multiple of \( ma \) and \( mb \), so that \( mh \mid H \). Also, \( H \) is a multiple of both \( ma \) and \( mb \), so \( H / m \) is a multiple of \( a \) and \( b \). Thus, \( H / m \mid h \), from which it follows that \( mh = H \), and this establishes the first part of the theorem.

    It will suffice to prove the second part for positive integers \( a \) and \( b
    \), since \( [a, -b] = [a, b] \). We begin with the special case where \( (a,
    b) = 1 \). Now \( [a, b] \) is a multiple of \( a \), say \( ma \). Then \( b
    \mid ma \) and \( (a, b) = 1 \), so by Theorem 1.10 we conclude that \( b \mid
    m \). Hence \( b \mid m \), \( ba \mid ma \). But \( ba \), being a positive
    common multiple of \( b \) and \( a \), cannot be less than the least common
    multiple, so \( ba = ma = [a, b] \).

    Turning to the general case where \( (a, b) = g > 1 \), we have \( (a/g, b/g) =
    1 \) by Theorem 1.7. Applying the result of the preceding paragraph, we obtain
    \[
        \left[ \frac{a}{g}, \frac{b}{g} \right] = \frac{ab}{g^2}.
    \]
    Multiplying by \( g^2 \) and using Theorem 1.6 as well as the first part of the
    present theorem, we get \( [a, b](a, b) = ab \).
\end{proof}

\section{January 17, 2025}
\begin{definition}
    An integer \(p > 1\) is called a \vocab{prime number} or \vocab{prime} in case there is no divisor of \(d\) of \(p\) satisfying \(1 < d < p\). An integer \(a > 1\) is not a prime, it is called \vocab{composite number}.
\end{definition}
\begin{example}
    \(2, 3, 5, 7\) are primes, but \(4, 6, 8, 9\) are composite.
\end{example}
\begin{theorem}
    Every integer \(n\) greater than \(1\) can be expressed as a product of primes.
\end{theorem}

\begin{proof}
    If the integer \(n\) is a prime, then the integer itself stands as a `product' with a single factor. Otherwise, \(n\) it can be factored into say \(n_1, n_2\), where \(1 < n_1 < n\) and \( 1 < n_2 < n\). If \(n_1\) is prime then let it stand. Otherwise, it will factor into say \(n_3, n4\) where \(1 < n_3 < n\) and \(1 < n_4 < n\). Simliarly, for \(n_2\). The process of writing each composite number that arises as a product of factors must termiate because the factors are smaller than the composite itself, yet each factor is an integer greater than 1. Thus we can conclude \(n\) as a product of \(q\) primes, and since the prime factors are not necessarily so the result can be written in the form
    \[ n = p_1^{\alpha_1}p_2^{\alpha_2}p_3^{\alpha_3}\ldots p_n^{\alpha_n}\]
    where the \(p_1, p_2, p_3, \ldots, p_n\) are distinct primes and \(\alpha_1,
    \alpha_2, \ldots, \alpha_n\) are positive
\end{proof}
\begin{fact}
    This representation of \(n\) as a product of primes is called the canonical
    factoring of \(n\) into prime numbers. It turns out that the representation is
    unique in the sense that, for \(a\) fixed \(n\) any other representation is
    merely a reordering, or a perumtation of factors, nevertheles it requires
    proof.
\end{fact}

\begin{theorem}
    If \(p \mid ab, p\) being a prime, then \(p \mid a\) or \(p \mid b\). More generally, if \(p \mid a_1 a_2\), then \(p\) at least one factor of \(a_1\).
\end{theorem}

\begin{proof}
    If \(p \nmid b\), since \((a, p) = 1\), by a previous theorem, \(p \mid b\). We may regard as a proof of the general cae of the statement mathematical induction. So we assume that the property holds when \(n\) divides a factor with fewer than \(n\) primes.
    Now, if \(p \mid a_1a_2\ldots a_n\), that is \(p \mid ac\), where \(c = a_1a_2\ldots a_n\), then \(p \mid a_1\) or \(p \mid c\).If \(p \mid c\), we apply the induction hypthesis to conclude that \(p \mid i\), for some subscript \(i = 1, 2, \ldots, n\).
\end{proof}

\begin{theorem}[The Fundamental Theorem of Arithmetic or the Unique Factorization Theorem]
    The facoring of \(n > 1\) into primes is unique and apart from the order of the primes.
\end{theorem}

\begin{proof}
    Suppose there is an integer \(n\) with two different factorizations. Dividing out any primes common to the two representations, we would have an equality of the form
    \[
        p_1 p_2 \cdots p_k = q_1 q_2 \cdots q_s
    \]
    where the factors \(p_i\) and \(q_j\) are primes, not necessarily all distinct,
    but where no prime on the left side occurs on the right side. But this is
    impossible because \(p_1 \mid q_1 q_2 \cdots q_s\), so by Theorem 1.15, \(p_1\)
    is a divisor of at least one of the \(q_j\). That is, \(p_1\) must be identical
    with at least one of the \(q_j\). This contradicts our assumption that no prime
    on the left side occurs on the right side. Therefore, the factorization of
    \(n\) into primes is unique.
\end{proof}

In the applications of the fundamental theorem, we frequently write the integer
\(a > \leqslant\) 1, in the form,
\[
    a = \prod_{i=1}^{n} p_i^{\alpha_i}
\]
where \(\alpha(p)\) is a non-negative integer for all sufficiently large
primes, \(p\). If \(a = 1\), then \(\alpha(p) = 0\), for all primes, \(p\) and
the product may be considered to be empty. We may write \(a = \prod
p^{\alpha}\)

It \(a = \prod_{p} p^{\alpha(p)}, b = \prod_{p} p^{\beta(p)}, c = \prod_{p}
p^{\beta(p)}\) and \(a = b = c\) then \(\alpha(p) + \beta(p) = \gamma(p)\) for
all \(p\). So, \(a \mid c\), we must note that \(\alpha(p) \leqslant
\gamma(p)\) for all \(p\) that we may define an integer \(b =
\prod_{p}p^{\beta(p)}\) with \(\beta = \gamma(p) - \alpha(p)\). So \(a \mid
c\). Note that the greatest common divisor and least common multiple can be
written as \[(a, b) = \prod_{p} p^{\min(\alpha(p), \beta(p))}\] \[[a, b] = \prod_{p} p^{\min(\alpha(p), \beta(p))}\]

\begin{example}
    \(a = 108, b = 225\), then \(a = 2^2 \cdot 3^3 \cdot 5^0\) and \(b = 2^0 \cdot 3^2 \cdot 5^2\). So \((a, b) = 2^0 \cdot 3^2 \cdot 5^0 = 9\), and \([a, b] = 2^2 \cdot 3^3 \cdot 5^2 = 2700\).
\end{example}

\begin{definition}
    \(a\) is a \vocab{square (or perfect square)} if it can be written as \(n^2\)
\end{definition}
\begin{remark}
    \(a\) is auare free if \(1\) is the largest square dividing \(a\). So \(\alpha(p)\) is square free if the only numbers are \(0\) and \(1\).
\end{remark}

\begin{theorem}[Euclid]
    THenumber of primes is inifite. i.e. there is no end to the sequence of primes. \[2, 3, 5, 7, 11, 13, \ldots\]
\end{theorem}

\begin{proof}
    Suppose that \(p_1, p_2, \ldots, p_n\) are the first \(r\) primes. Then form the number \[n = 1 + p_1 p_2\ldots p_r\]
    Note that \(n\) is not divisible by \(p_1\) or \(p_2\) or \(\ldots\), or
    \(p_r\) Hencce, ayny prime divisor is distinct from \(p_1, p2, \ldots, p_r\).
    Since \(n\) is neither a prime or has a prime factor factor \(p\). This impl
\end{proof}
%\section{January 20, 2025}
%\section{January 22, 2025}
%\section{January 24, 2025}
%\section{January 27, 2025}
%\section{January 29, 2025}
%\section{January 31, 2025}
%\section{February 3, 2025}
%\section{February 5, 2025}
%\section{February 7, 2025}
%\section{February 10, 2025}
%\section{February 12, 2025}
%\section{February 14, 2025}
%\section{February 17, 2025}
%\section{February 19, 2025}
%\section{February 21, 2025}
%\section{February 24, 2025}
%\section{February 26, 2025}
%\section{February 28, 2025}
%\section{March 3, 2025}
%\section{March 5, 2025}
%\section{March 7, 2025}
%\section{March 17, 2025}
%\section{March 19, 2025}
%\section{March 21, 2025}
%\section{March 24, 2025}
%\section{March 26, 2025}
%\section{March 28, 2025}
%\section{March 31, 2025}
%\section{April 2, 2025}
%\section{April 4, 2025}
%\section{April 7, 2025}
%\section{April 9, 2025}
%\section{April 11, 2025}
%\section{April 14, 2025}
%\section{April 16, 2025}
%\section{April 18, 2025}
%\section{April 21, 2025}
%\section{April 23, 2025}
%\section{April 25, 2025}
%\section{April 28, 2025}

\end{document}
