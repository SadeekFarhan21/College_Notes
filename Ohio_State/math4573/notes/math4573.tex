% !TEX TS-program = lualatex
%\documentclass[11pt, draft]{article}
\documentclass[11pt]{article}
\usepackage{lindrew}
\usepackage{xcolor}
\usepackage{amsmath}
\usepackage{amssymb}
\usepackage{tikz}
\usepackage{hyperref}
\usepackage{fontspec}
\usepackage{cleveref}
\title{Math 4573: Number Theory}
\author{Lecturer: \textbf{Professor James Cogdell}\\Notes by: Farhan Sadeek}
\date{Spring 2025}

\begin{document}
\maketitle

%%%%%%%%%%%%%%%%%%%%%%%%%%%%
\section{January 8, 2025}

Dr.\ Cogdell explained the logistics of the class and also took attendance.
This class will be no exams and graded based on only homeworks.

\subsection{Conjectures in Number Theory}
\begin{itemize}
    \item A number is divisible by 3 if the sum of its digits is divisible by 3.
    \item \textbf{Fermat's Last Theorem}: There are no three positive integers $a$, $b$, and $c$ that satisfy the equation $a^n + b^n = c^n$ for any integer value of $n$ greater than 2.
    \item There are infinitely many primes.
    \item $\sqrt{2}$ is irrational.
    \item $\pi$ is irrational.
    \item Every number can be written as the sum of four squares (Lagrange's Four Square
          Theorem). For example, $1000 = 10^2 + 30^2 + 0^2 + 0^2$ and $999 = 30^2 + 9^2 +
              3^2 + 3^2$.
    \item The polynomial $n^2 - n + 41$ produces prime numbers for $n = 0, 1, 2, \ldots,
              40$, but not for $n = 41$.
    \item Euler conjectured that no $n^{th}$ power can be written as the sum of two
          $n^{th}$ powers for $n > 2$. This was proven false by the counterexample $144^5
              = 27^5 + 84^5 + 110^5 + 133^5$.
    \item \textbf{Goldbach's Conjecture}: Every even integer greater than 2 can be written as the sum of two primes. For example, $4 = 2 + 2$, $6 = 3 + 3$, $8 = 3 + 5$, $10 = 5 + 5$, $12 = 5 + 7$, $14 = 7 + 7$, $16 = 3 + 13$, $18 = 7 + 11$. This has been verified for numbers up to 100,000 but remains unproven.
\end{itemize}

Number theory is related to \textbf{Abstract Algebra}, but also intersects with
other domains such as \textbf{Combinatorics, Analysis, and Topology}. We will
accept a few fundamental facts about \textbf{Number Theory}.

\begin{fact}
    If \(\mathcal{S}\) is a non-empty set of positive integers, then \(\mathcal{S}\) contains a smallest element. This is known as the Well-Ordering Principle.
\end{fact}

\subsection{Divisibility}
This concept has been known since the time of Euclid.

\begin{definition}
    An integer $b$ is divisible by an integer $a \neq 0$ if there is an integer $x$ such that $b = ax$. We write this as $a \mid b$. If $b$ is not divisible by $a$, we write $a \nmid b$.
\end{definition}

There are two derivative notions:
\begin{itemize}
    \item If $0 < a < b$, then $a$ is called a \vocab{proper divisor} of $b$.
    \item If $a^k \mid\mid b$, it means $a^k \mid b$ and $a^{k + 1} \nmid b$.
\end{itemize}

\begin{theorem}\label{1.1}
    Let $a$, $b$, and $c$ be integers. Then the following are true:
    \begin{itemize}
        \item If $a \mid b$, then $a \mid bc$.
        \item If $a \mid b$, then $a \mid b + c$.
        \item If $a \mid b$ and $a \mid c$, then $a \mid b + c$.
        \item If $a \mid b$ and $b \mid a$, then $a = b$ or $a = -b$.
        \item If $a \mid b$ and $a > 0$ and $b > 0$, then $a \leq b$.
        \item If $m \neq 0$ and $a \mid b$, then $am \mid bm$.
        \item If $a \mid b_1, a \mid b_2, \ldots, a \mid b_n$, then $a \mid \sum_{i=1}^{n}
                  b_i x_i$ for any integers $x_i$.
    \end{itemize}
\end{theorem}

\begin{theorem}[The Division Algorithm]\label{1.2}
    Given integers \(a\) and \(b\) with \(a > 0\), there exist unique integers \(q\) and \(r\) such that
    \[
        b = qa + r, \quad 0 \leq r < a.
    \]
    If \(a \nmid b\), then \(r\) satisfies the stronger inequality
    \[
        0 < r < a.
    \]
\end{theorem}

\begin{proof}
    Consider the arithmetic progression $\ldots, b - 3a, b - 2a, b - a, b, b + a, b + 2a, b + 3a, \ldots$. In this sequence, select the smallest non-negative member. This defines $r$ and satisfies the inequalities of the theorem. Since $r$ is in the sequence, it can be written as $b - qa$. To prove the uniqueness of $q$ and $r$, suppose there is another pair $q_1$ and $r_1$ that satisfies the same conditions. We first prove that $r = r_1$. If not, assume $r < r_1$, so $0 < r_1 - r < a$. But $r_1 - r = a(q - q_1)$, meaning $a \mid (r_1 - r)$, which contradicts the fact that $0 < r_1 - r < a$. Thus, $r = r_1$ and $q = q_1$.
\end{proof}

\begin{fact}
    If $a \mid b$, then $r$ satisfies the stronger inequality $0 \leq r < a$.
\end{fact}

\begin{fact}
    The Division Algorithm can be stated without the assumption $a > 0$. Given integers $a$ and $b$ with $a \neq 0$, there exist integers $q$ and $r$ such that $b = qa + r$ with $0 \leq |r| < |a|$.
\end{fact}

\begin{definition}[Common Divisor]
    The integer $a$ is a \vocab{common divisor} of $b$ and $c$ if $a \mid b$ and $a \mid c$. Since there is only a finite number of divisors of any non-zero integer, there is only a finite number of common divisors of $b$ and $c$ except in the case $b = c = 0$.
\end{definition}

If at least one of \( b \) and \( c \) is not \( 0 \), the \vocab{greatest
    common divisor} is called the \vocab{gcd} \( \text{gcd}(b, c) \)
(\textit{greatest common divisor of \( b \) and \( c \)}), and is denoted by \(
(b, c) \). Similarly, we have the greatest common divisor \( g \) of the
integers \( b_1, b_2, \ldots, b_n \) (\textit{not all \( 0 \)}) denoted by \(
(b_1, b_2, \ldots, b_n) \).

\begin{theorem}\label{1.3}
    If $g$ is the \vocab{gcd} of \( b \) and \( c \), then there exist integers \( x_0 \) and \( y_0 \) such that \[g = bx_0 + cy_0\]
\end{theorem}

\section{January 10, 2025}
Dr.\ Cogdell takes attendance so I will have to be in class every single day.

\begin{definition}[Common Divisor]
    The integer $a$ is a common divisor of $b$ and $c$ if $a \mid b$ and $a \mid c$. Since there is only a finite number of divisors of any nonzero integer, there is only a finite number of common divisors of $b$ and $c$, except in the case $b = c = 0$. If at least one of $b$ and $c$ is not $0$, the greatest among their common divisors is called the greatest common divisor of $b$ and $c$ and is denoted by $(b, c)$. Similarly, we denote the greatest common divisor $g$ of the integers $b_1, b_2, \ldots, b_n$, not all zero, by $(b_1, b_2, \ldots, b_n)$.
\end{definition}

\begin{fact}
    Another fundamental way to to state this is that the linear combination of \( b \) and \( c \) is with integgral multipliers \( x_0 \) and \( y_0 \). This assertion of holds for any finite collection.
\end{fact}
\begin{proof}
    Consider the following linear combinations \{\(bx + cy\)\} where \(x\) and \(y\) are all integers. Note this also contains \(x = y = 0\). Choose \(bx_0 + cy_0 \) is the least positive integer \(l\) in the set.

    We need to prove that \(l \mid b\) and \(l \mid c\). We will do this via
    indirect proof. If we assume that \(l \nmid b\), we will obtain a
    contradiction. From \(l \nmid b\), there are integers \(q\) and \(r\) such that
    \(b = lq + r\) where \(0 < r < l\). Since \(l\) is the least positive integer
    in the set, we can write \(r = bx_1 + cy_1\) for some integers \(x_1\) and
    \(y_1\). So we have \[r = b - lq = b - q(bx_0 - cy_0) = b(1 - qx_0) + c(-qy_0)\]  and this \(r\) is in the set {\(bx + cy\)}. This contradicts the fact that
    \(l\) is the least positive integer in the set \{\(bx + cy\)\}. Thus, we have
    shown that \(l \mid b\).

    Since \(g\) is the greatest common divisor of \(b\) and \(c\), we may write \(l
    = bx_0 + cy_0 = g(Bx_0 + Cy_0)\). Then, \(g \mid l\) and we have shown \(g \leq
    l\). Now, \(g < l\) is impossible since, \(g\)is the greatest common divisor,
    so \(g = l = bx_0 + cy_0\).
\end{proof}

\begin{theorem}\label{1.4}
    The greatest common divisor \(g\) of \(b\) and \(c\) can be characterized in the following two ways:
    \begin{itemize}
        \item It is the least positive value of \(bx + cy\) where \(x\) and \(y\) range over
              all integers.
        \item It is the positive common divisor of \(b\) and \(c\) that is divisible by every
              common divisor.
    \end{itemize}
\end{theorem}

\begin{proof}
    Part 1 follows from the proof of \cref{1.3}. To prove part 2, we observe that if \(d\) is any common divisor of \(b\) and \(c\), then \(d \mid g\) by part 3 of \cref{1.1}. Moreover, there cannot be two distinct integers with property 2, because of \cref{1.1}, part 4.
\end{proof}

\begin{remark}
    If an integer \(d\) is expressible in the form \(d = bx + cy\), then \(d\) is not
    necessarily the \(\gcd(b, c)\). However, it does follow from such an equation
    that \((b, c)\) is a divisor of \(d\). In particular, if \(bx + cy = 1\) for some integers
    \(x\) and \(y\), then \((b, c) = 1\).
\end{remark}

\begin{theorem}\label{1.5}
    Given any integers $b_1, b_2, \ldots, b_n$ not all zero, with greatest common divisor $g$, there exist integers $x_1, x_2, \ldots, x_n$ such that
    \[
        g = (b_1, b_2, \ldots, b_n) = \sum_{j=1}^{n} b_j x_j.
    \]
    Furthermore, $g$ is the least positive value of the linear form $\sum_{j=1}^{n}
        b_j y_j$ where the $y_j$ range over all integers; also $g$ is the positive
    common divisor of $b_1, b_2, \ldots, b_n$ that is divisible by every common
    divisor.
\end{theorem}
\begin{proof}
    Consider the set \( S = \left\{ \sum_{j=1}^{n} b_j y_j \mid y_j \in \mathbb{Z} \right\} \). Since not all \( b_j \) are zero, there exists a non-zero integer in \( S \). Let \( g \) be the smallest positive integer in \( S \). Then \( g \) can be written as \( g = \sum_{j=1}^{n} b_j x_j \) for some integers \( x_j \).

    We claim that \( g \) is the greatest common divisor of \( b_1, b_2, \ldots,
    b_n \). First, we show that \( g \) is a common divisor of \( b_1, b_2, \ldots,
    b_n \). For each \( b_i \), we have
    \[
        b_i = \sum_{j=1}^{n} b_j \delta_{ij},
    \]
    where \( \delta_{ij} \) is the Kronecker delta. Since \( g \) divides each term
    on the right-hand side, it follows that \( g \mid b_i \) for all \( i \).

    Next, we show that \( g \) is the greatest common divisor. Let \( d \) be any
    common divisor of \( b_1, b_2, \ldots, b_n \). Then \( d \mid \sum_{j=1}^{n}
    b_j x_j \), so \( d \mid g \). Therefore, \( g \) is the greatest common
    divisor of \( b_1, b_2, \ldots, b_n \).

    Finally, we show that \( g \) is the least positive value of the linear form \(
    \sum_{j=1}^{n} b_j y_j \). Suppose there exists a positive integer \( h \) such
    that \( h = \sum_{j=1}^{n} b_j z_j \) and \( h < g \). Then \( h \) is in \( S
    \), which contradicts the minimality of \( g \). Therefore, \( g \) is the
    least positive value of the linear form.

    Thus, we have shown that \( g = (b_1, b_2, \ldots, b_n) = \sum_{j=1}^{n} b_j
    x_j \) and \( g \) is the least positive value of the linear form \(
    \sum_{j=1}^{n} b_j y_j \) where the \( y_j \) range over all integers. Also, \(
    g \) is the positive common divisor of \( b_1, b_2, \ldots, b_n \) that is
    divisible by every common divisor.
\end{proof}

\begin{theorem}\label{1.6}
    For any positive integer\(m\) we have \[(ma, mb) = m(a, b)\]
\end{theorem}

\begin{proof}
    By \cref{1.4} we have
    \begin{align*}
        (ma, mb) & = \text{least positive value of } max + mby           \\
                 & = m \cdot \{\text{least positive value of } ax + by\} \\
                 & = m(a, b).
    \end{align*}
\end{proof}
\begin{theorem}\label{1.7}
    If \(d \mid a\) and \(d \mid b\), \(d > 0\), then \[(\frac{a}{b}, \frac{b}{d} ) = \frac{1}{d}(a, b)\]
    If (\(a, b) = g\), then \[(\frac{a}{g}, \frac{b}{g}) = 1\]
\end{theorem}

\begin{proof}
    The second assertion is the special case of the first obtained by using the greatest common divisor \(g\) of \(a\) and \(b\) in the role of \(d\). The first assertion in turn is a direct consequence of \cref{1.6} obtained by replacing \(m, a, b\) in that theorem by \(d, \frac{a}{d}, \frac{b}{d}\) respectively.
\end{proof}

\begin{theorem}\label{1.8}
    If (\(a, m\)) = (\(b, m\)) = 1, then (\(ab, m\)) = 1
\end{theorem}
\begin{proof}
    By \cref{1.3}, there exist integers \(x_0, y_0, x_1, y_1\) such that
    \[1 = ax_0 + my_0 = bx_1 + my_1.\]
    Thus, we may write
    \[
        a x_0 - b x_1 = m (y_1 - y_0).
    \]
    Let \(y_2 = y_1 - y_0\). Then we have
    \[
        a x_0 - b x_1 = m y_2.
    \]
    From the equation \(a x_0 - b x_1 = m y_2\), we note, by part 3 \cref{1.1},
    that any common divisor of \(a\) and \(b\) is a divisor of \(m\). Hence, \((a,
    b, m) = 1\).
\end{proof}
\begin{theorem}\label{1.9}
    For any integers \(a\) and \(b\), the following equalities hold:
    \[
        (a, b) = (b, a) = (a, -b) = (a, b + ax).
    \]
\end{theorem}
\begin{proof}
    The equality \((a, b) = (b, a)\) follows from the definition of the greatest common divisor, as the order of the arguments does not affect the set of common divisors.

    The equality \((a, b) = (a, -b)\) holds because the set of common divisors of
    \(a\) and \(b\) is the same as the set of common divisors of \(a\) and \(-b\).

    To prove \((a, b) = (a, b + ax)\), we note that any common divisor of \(a\) and
    \(b\) is also a divisor of \(b + ax\) (since \(b + ax = b + a \cdot x\)).
    Conversely, any common divisor of \(a\) and \(b + ax\) is also a divisor of
    \(b\) (since \(b = (b + ax) - a \cdot x\)). Therefore, the set of common
    divisors of \(a\) and \(b\) is the same as the set of common divisors of \(a\)
    and \(b + ax\), which implies that \((a, b) = (a, b + ax)\).
\end{proof}
\begin{theorem}\label{1.10}
    If \(c \mid ab\) and \((b, c) = 1\), then \(c \mid a\).
\end{theorem}

\begin{proof}
    Since \((b, c) = 1\), there exist integers \(x\) and \(y\) such that \(bx + cy = 1\). Multiplying both sides by \(a\), we get
    \[
        abx + acy = a.
    \]
    Since \(c \mid ab\), there exists an integer \(k\) such that \(ab = ck\).
    Substituting this into the equation, we get
    \[
        ckx + acy = a.
    \]
    Factoring out \(c\) from the left-hand side, we get
    \[
        c(kx + ay) = a.
    \]
    Therefore, \(c \mid a\).
\end{proof}
\section{January 13, 2025}
\subsection{Euclidean Algorithm}
Given two integers \(b\) and \(c\), now we can generate the greatest common
divisor. There is no algorithm to this problem, but there is an algorithm.

\begin{question}
    Given a set of integers (\(bx + cy\)) how to find the greatest common divisor?
\end{question}
Consider the case \(b = 963\) and \(c = 657\). If we divide \(c\) into \(b\), we get the quotient \(q = 1\) and the remainder \(r = 306\). We can write this as \(b = qc + r\) or \(r = b - cq\). In particular, \(306 = 963 - 1 \cdot 657\). Now \((b, c) = (b - cq, c)\) by replacing \(a\) and \(x\) by \(c\) and \(-q\) in \cref{1.9}, so we see that
\[
    (963, 657) = (963 - 1 \cdot 657, 657) = (306, 657).
\]
The integer 963 has been replaced by the smaller integer 306, and this suggests
that the procedure be repeated. So we divide 306 into 657 to get a quotient 2
and a remainder 45, and
\[
    (306, 657) = (306, 657 - 2 \cdot 306) = (306, 45).
\]
Next, 45 is divided into 306 with quotient 6 and remainder 36, then 36 is
divided into 45 with quotient 1 and remainder 9. We conclude that
\[
    (963, 657) = (306, 657) = (306, 45) = (45, 36) = (36, 9).
\]
Thus \((963, 657) = 9\), and we can express 9 as a linear combination of 963
and 657 by sequentially writing each remainder as a linear combination of the
two original numbers:
\[
    \begin{aligned}
        306 & = 963 - 657,                                                                                          \\
        45  & = 657 - 2 \cdot 306 = 657 - 2 \cdot (963 - 657) = 3 \cdot 657 - 2 \cdot 963,                          \\
        36  & = 306 - 6 \cdot 45 = (963 - 657) - 6 \cdot (3 \cdot 657 - 2 \cdot 963) = 13 \cdot 963 - 19 \cdot 657, \\
        9   & = 45 - 36 = 3 \cdot 657 - 2 \cdot 963 - (13 \cdot 963 - 19 \cdot 657) = 22 \cdot 657 - 15 \cdot 963.
    \end{aligned}
\]
In terms of \cref{1.3}, where \(g = (b, c) = bx_0 + cy_0\), beginning with \(b
= 963\) and \(c = 657\) we have used a procedure called the Euclidean algorithm
to find \(g = 9\), \(x_0 = -15\), \(y_0 = 22\). Of course, these values for
\(x_0\) and \(y_0\) are not unique: \(-15 + 657k\) and \(22 - 963k\) will do
where \(k\) is any integer.

To find the greatest common divisor \((b, c)\) of any two integers \(b\) and
\(c\), we now generalize what is done in the special case above. The process
will also give integers \(x_0\) and \(y_0\) satisfying the equation \(bx_0 +
cy_0 = (b, c)\). The case \(c = 0\) is special: \((b, 0) = |b|\). For \(c \neq
0\), we observe that \((b, c) = (b, -c)\) by \cref{1.9}, and hence, we may
presume that \(c\) is positive.

\begin{theorem}[The Euclidean Algorithm]\label{1.11}
    Given integers \( b \) and \( c > 0 \), we make a repeated application of the division algorithm, \cref{1.2}, to obtain a series of equations:
    \[
        \begin{aligned}
            b       & = cq_1 + r_1, \quad 0 < r_1 < c,                                         \\
            c       & = r_1q_2 + r_2, \quad 0 < r_2 < r_1,                                     \\
            r_1     & = r_2q_3 + r_3, \quad 0 < r_3 < r_2,                                     \\
                    & \ \, \vdots \quad \quad \quad \quad \quad \quad \quad \quad \quad \vdots \\
            r_{j-2} & = r_{j-1}q_j + r_j, \quad 0 < r_j < r_{j-1},                             \\
            r_{j-1} & = r_jq_{j+1}.
        \end{aligned}
    \]

    The greatest common divisor \((b, c)\) of \( b \) and \( c \) is \( r_j \), the
    last nonzero remainder in the division process. Values of \( x_0 \) and \( y_0
    \) in \((b, c) = bx_0 + cy_0 \) can be obtained by writing each \( r_i \) as a
    linear combination of \( b \) and \( c \).
\end{theorem}

\begin{proof}
    The chain of equations is obtained by dividing \(c\) into \(b\), \(r_1\) into \(c\), \(r_2\) into \(r_1\), and so on, until \(r_j\) into \(r_{j-1}\). The process stops when the division is exact, that is, when the remainder is zero. Thus, in our application of \cref{1.2}, we have written the inequalities for the remainder without an equality sign. For example, \(0 < r_1 < c\) instead of \(0 \leq r_1 < c\), because if \(r_1\) were equal to zero, the chain would stop at the first equation \(b = cq_1\), in which case the greatest common divisor of \(b\) and \(c\) would be \(c\).

    We now prove that \(r_j\) is the greatest common divisor \(g\) of \(b\) and
    \(c\). By \cref{1.9}, we observe that
    \[
        (b, c) = (c, r_1) = (r_1, r_2) = \cdots = (r_{j-1}, r_j) = (r_j, 0) = r_j.
    \]

    To see that \(r_j\) is a linear combination of \(b\) and \(c\), we argue by
    induction that each \(r_i\) is a linear combination of \(b\) and \(c\).
    Clearly, \(r_1\) is such a linear combination, and likewise \(r_2\). In
    general, \(r_i\) is a linear combination of \(r_{i-1}\) and \(r_{i-2}\). By the
    inductive hypothesis, we may suppose that these latter two numbers are linear
    combinations of \(b\) and \(c\), and it follows that \(r_i\) is also a linear
    combination of \(b\) and \(c\).
\end{proof}

\section{January 15, 2025}
\begin{example}
    We will find the g.c.d of 42823 and 6409.
\end{example}
\begin{solution}

    We apply the Euclidean algorithm to divide \(c\) into \(b\), where \(b =
    42823\) and \(c = 6409\). We obtain a quotient \(q_1 = 6\) and a remainder
    \(r_1 = 4369\). Continuing, if we divide 4369 into 6409, we get a quotient
    \(q_2 = 1\) and a remainder \(r_2 = 2040\). Dividing 2040 into 4369 gives \(q_3
    = 2\) and \(r_3 = 289\). Dividing 289 into 2040 gives \(q_4 = 7\) and \(r_4 =
    17\). Since 17 is an exact divisor of 289, the solution is that the g.c.d is
    17.

    We can write this in tabular form:
    \[
        \begin{aligned}
            42823 & = 6 \cdot 6409 + 4369, \\
            6409  & = 1 \cdot 4369 + 2040, \\
            4369  & = 2 \cdot 2040 + 289,  \\
            2040  & = 7 \cdot 289 + 17,    \\
            289   & = 17 \cdot 17.
        \end{aligned}
    \]
    Thus, \((42823, 6409) = (6409, 4369) = (4369, 2040) = (2040, 289) = (289, 17) =
    17\).
\end{solution}

\begin{example}
    Find integers \(x\) and \(y\) such that \(42823x + 6409y = 17\).
\end{example}
\begin{solution}

    We find integers \(x\) and \(y\) such that \(42823x + 6409y = 17\).

    Here it is natural to consider \(i = 1, 2, \ldots\), but to initiate the
    process we also consider \(i = 0\) and \(i = -1\). We put \(r_{-1} = 42823\),
    and write
    \[42823 \cdot 1 + 6409 \cdot 0 = 42823.\]
    Similarly, we put \(r_0 = 6409\), and write
    \[42823 \cdot 0 + 6409 \cdot 1 = 6409.\]
    We multiply the second of these equations by \(q_1 = 6\), and subtract the
    result from the first equation, to obtain
    \[42823 \cdot 1 + 6409 \cdot (-6) = 4369.\]
    We multiply this equation by \(q_2 = 1\), and subtract it from the preceding
    equation to find that
    \[42823 \cdot (-1) + 6409 \cdot 7 = 2040.\]
    We multiply this by \(q_3 = 2\), and subtract the result from the preceding
    equation to find that
    \[42823 \cdot 3 + 6409 \cdot (-20) = 289.\]
    Next we multiply this by \(q_4 = 7\), and subtract the result from the
    preceding equation to find that
    \[42823 \cdot (-22) + 6409 \cdot 147 = 17.\]
    On dividing 17 into 289, we find that \(q_5 = 17\) and that \(289 = 17 \cdot
    17\). Thus \(r_4\) is the last positive remainder, so that \(g = 17\), and we
    may take \(x = -22\), \(y = 147\). These values of \(x\) and \(y\) are not the
    only ones possible. In Section 5.1, an analysis of all solutions of a linear
    equation is given.
\end{solution}

\begin{remark}
    Section 5.1 on Analysis
\end{remark}

\begin{definition}
    The integers \(a_1, a_2, \ldots , a_n\) all different from zero, have a common \(b\) if \(a_i \mid b\) for \(i = 1, 2, \ldots, n\). The least positive multiple is is called \vocab{least common multiple} and it's denoted \([a_1, a_2, \ldots, a_n]\)
\end{definition}

\begin{theorem}\label{1.12}
    If \(b\) is any common multiple of \(a_1, a_2, \ldots, a_n\), then \([a_1, a_2, \ldots, a_n] \mid b\). This is the same as saying that if \(h\) denotes \([a_1, a_2, \ldots, a_n]\), then \(0, \pm h, \pm 2h, \pm 3, \ldots\) comprise all the common multiples of \(a_1, a_2, \ldots, a_n\).
\end{theorem}
\begin{proof}
    Let \(m\) be any common multiple and divide \(m\) by \(h\). By Division Algorithm, there is a quotient \(q\) and a remainder \(r\) such that \(m = qh + r\), where \(0 \leqslant r < h\). We must prove that \(r = 0\). If \(r \neq -\), we argue as follows. For each \(i = 1, 2, \ldots, n\), we know that \(a_i \mid h\) and \(a_i \mid m\), so that \(a_i \mid r\). Thus \(r\) is a positive common multiple of \(a_1, a_2, \ldots, a_n\) contrary to the fact that \(h\) is the least of all common positive multiple.
\end{proof}

\begin{theorem}\label{1.13}
    If \(m > 0\) \([ma, mb] = m[a, b]\). Also, \([a,b] \cdot (a, b) = |ab|\)
\end{theorem}
\begin{proof}
    Let \( H = [ma, mb] \) and \( h = [a, b] \). Then \( mh \) is a multiple of \( ma \) and \( mb \), so that \( mh \mid H \). Also, \( H \) is a multiple of both \( ma \) and \( mb \), so \( H / m \) is a multiple of \( a \) and \( b \). Thus, \( H / m \mid h \), from which it follows that \( mh = H \), and this establishes the first part of the theorem.

    It will suffice to prove the second part for positive integers \( a \) and \( b
    \), since \( [a, -b] = [a, b] \). We begin with the special case where \( (a,
    b) = 1 \). Now \( [a, b] \) is a multiple of \( a \), say \( ma \). Then \( b
    \mid ma \) and \( (a, b) = 1 \), so by \cref{1.10} we conclude that \( b \mid m
    \). Hence \( b \mid m \), \( ba \mid ma \). But \( ba \), being a positive
    common multiple of \( b \) and \( a \), cannot be less than the least common
    multiple, so \( ba = ma = [a, b] \).

    Turning to the general case where \( (a, b) = g > 1 \), we have \( (a/g, b/g) =
    1 \) by \cref{1.7}. Applying the result of the preceding paragraph, we obtain
    \[
        \left[ \frac{a}{g}, \frac{b}{g} \right] = \frac{ab}{g^2}.
    \]
    Multiplying by \( g^2 \) and using \cref{1.6} as well as the first part of the
    present theorem, we get \( [a, b](a, b) = ab \).
\end{proof}

\section{January 17, 2025}
\begin{definition}
    An integer \(p > 1\) is called a \vocab{prime number} or \vocab{prime} in case there is no divisor of \(d\) of \(p\) satisfying \(1 < d < p\). An integer \(a > 1\) is not a prime, it is called \vocab{composite number}.
\end{definition}
\begin{example}
    \(2, 3, 5, 7\) are primes, but \(4, 6, 8, 9\) are composite.
\end{example}
\begin{theorem}\label{1.14}
    Every integer \(n\) greater than \(1\) can be expressed as a product of primes.
\end{theorem}

\begin{proof}
    If the integer \(n\) is a prime, then the integer itself stands as a `product' with a single factor. Otherwise, \(n\) it can be factored into say \(n_1, n_2\), where \(1 < n_1 < n\) and \( 1 < n_2 < n\). If \(n_1\) is prime then let it stand. Otherwise, it will factor into say \(n_3, n4\) where \(1 < n_3 < n\) and \(1 < n_4 < n\). Simliarly, for \(n_2\). The process of writing each composite number that arises as a product of factors must termiate because the factors are smaller than the composite itself, yet each factor is an integer greater than 1. Thus we can conclude \(n\) as a product of \(q\) primes, and since the prime factors are not necessarily so the result can be written in the form
    \[ n = p_1^{\alpha_1}p_2^{\alpha_2}p_3^{\alpha_3}\ldots p_n^{\alpha_n}\]
    where the \(p_1, p_2, p_3, \ldots, p_n\) are distinct primes and \(\alpha_1,
    \alpha_2, \ldots, \alpha_n\) are positive
\end{proof}
\begin{fact}
    This representation of \(n\) as a product of primes is called the canonical
    factoring of \(n\) into prime numbers. It turns out that the representation is
    unique in the sense that, for \(a\) fixed \(n\) any other representation is
    merely a reordering, or a perumtation of factors, nevertheles it requires
    proof.
\end{fact}

\begin{theorem}\label{1.15}
    If \(p \mid ab, p\) being a prime, then \(p \mid a\) or \(p \mid b\). More generally, if \(p \mid a_1 a_2\), then \(p\) at least one factor of \(a_1\).
\end{theorem}

\begin{proof}
    If \(p \nmid b\), since \((a, p) = 1\), by a previous theorem, \(p \mid b\). We may regard as a proof of the general cae of the statement mathematical induction. So we assume that the property holds when \(n\) divides a factor with fewer than \(n\) primes.
    Now, if \(p \mid a_1a_2\ldots a_n\), that is \(p \mid ac\), where \(c = a_1a_2\ldots a_n\), then \(p \mid a_1\) or \(p \mid c\).If \(p \mid c\), we apply the induction hypthesis to conclude that \(p \mid i\), for some subscript \(i = 1, 2, \ldots, n\).
\end{proof}

\begin{theorem}[The Fundamental Theorem of Arithmetic or the Unique Factorization Theorem]\label{1.16}
    The facoring of \(n > 1\) into primes is unique and apart from the order of the primes.
\end{theorem}

\begin{proof}
    Suppose there is an integer \(n\) with two different factorizations. Dividing out any primes common to the two representations, we would have an equality of the form
    \[
        p_1 p_2 \cdots p_k = q_1 q_2 \cdots q_s
    \]
    where the factors \(p_i\) and \(q_j\) are primes, not necessarily all distinct,
    but where no prime on the left side occurs on the right side. But this is
    impossible because \(p_1 \mid q_1 q_2 \cdots q_s\), so by \cref{1.15}, \(p_1\)
    is a divisor of at least one of the \(q_j\). That is, \(p_1\) must be identical
    with at least one of the \(q_j\). This contradicts our assumption that no prime
    on the left side occurs on the right side. Therefore, the factorization of
    \(n\) into primes is unique.
\end{proof}

In the applications of the fundamental theorem, we frequently write the integer
\(a > \leqslant\) 1, in the form,
\[
    a = \prod_{i=1}^{n} p_i^{\alpha_i}
\]
where \(\alpha(p)\) is a non-negative integer for all sufficiently large
primes, \(p\). If \(a = 1\), then \(\alpha(p) = 0\), for all primes, \(p\) and
the product may be considered to be empty. We may write \(a = \prod
p^{\alpha}\)

It \(a = \prod_{p} p^{\alpha(p)}, b = \prod_{p} p^{\beta(p)}, c = \prod_{p}
p^{\beta(p)}\) and \(a = b = c\) then \(\alpha(p) + \beta(p) = \gamma(p)\) for
all \(p\). So, \(a \mid c\), we must note that \(\alpha(p) \leqslant
\gamma(p)\) for all \(p\) that we may define an integer \(b =
\prod_{p}p^{\beta(p)}\) with \(\beta = \gamma(p) - \alpha(p)\). So \(a \mid
c\). Note that the greatest common divisor and least common multiple can be
written as \[(a, b) = \prod_{p} p^{\min(\alpha(p), \beta(p))}\] \[[a, b] = \prod_{p} p^{\min(\alpha(p), \beta(p))}\]

\begin{example}
    \(a = 108, b = 225\), then \(a = 2^2 \cdot 3^3 \cdot 5^0\) and \(b = 2^0 \cdot 3^2 \cdot 5^2\). So \((a, b) = 2^0 \cdot 3^2 \cdot 5^0 = 9\), and \([a, b] = 2^2 \cdot 3^3 \cdot 5^2 = 2700\).
\end{example}

\begin{definition}
    \(a\) is a \vocab{square (or perfect square)} if it can be written as \(n^2\)
\end{definition}
\begin{remark}
    \(a\) is auare free if \(1\) is the largest square dividing \(a\). So \(\alpha(p)\) is square free if the only numbers are \(0\) and \(1\).
\end{remark}

\begin{theorem}[Euclid]\label{1.17}
    The number of primes is inifite. i.e. there is no end to the sequence of primes. \[2, 3, 5, 7, 11, 13, \ldots\]
\end{theorem}

\begin{proof}
    Suppose that \(p_1, p_2, \ldots, p_n\) are the first \(r\) primes. Then form the number \[n = 1 + p_1 p_2\ldots p_r\]
    Note that \(n\) is not divisible by \(p_1\) or \(p_2\) or \(\ldots\), or
    \(p_r\) Hencce, ayny prime divisor is distinct from \(p_1, p2, \ldots, p_r\).
    Since \(n\) is neither a prime or has a prime factor factor \(p\). This impl
\end{proof}
%\section{January 20, 2025}
\section{January 22, 2025}
\begin{theorem}\label{1.18}
    There are arbitrarily large gapes in the series of primes stated otherwise, given any \(k\), there exit \(k\) consequetive composite integers.
\end{theorem}
\begin{proof}
    Consider the integers
    \[(k + 1)! + 2, (k + 1)! + 3 \ldots, (k + 1)! + k, (k + 1)! + k + 1\]
    Every one of these composite because \(j\) divides \((k + 1)!\) and \(j
    \leqslant k\).
\end{proof}

The primes are spaced rather irregularly, as the last theorem suggests. If we
denote the number of pirmes that do not exceed \(x\) by \(\pi(x)\), but we may
ask about the nature of this function. Because of this irregular occurence of
primes, we cannnot expect a simple formula for \(\pi(x)\), but we may week to
estimate the rate of it's growth.

\begin{theorem}\label{1.19}
    For any real number \(y \geqslant 2\), we have
    \[\sum_{p \leqslant y} \frac{1}{p} \log \log y - 1\]
\end{theorem}

\subsection{The Binomial Theorem}
We first define the \texttt{binomial coefficients} and describe them
combinatorially.

\begin{definition}
    Let \(\alpha\) be any real number, and let \(k\) be a non-negative integer. Then the binomial coefficient \(\binom{\alpha}{k}\) is given by the formula:
    \[
        \binom{\alpha}{k} = \frac{\alpha(\alpha - 1)(\alpha - 2) \cdots (\alpha - k + 1)}{k!}
    \]
    Suppose that \(n\) and \(k\) are both integers. From the formula, we see that
    if \(0 \leq k \leq n\), then
    \[
        \binom{n}{k} = \frac{n!}{k!(n - k)!},
    \]
    whereas if \(n < k\), then
    \[
        \binom{n}{k} = 0.
    \]
    Here we employ the convention \(0! = 1\).
\end{definition}

\begin{theorem}\label{1.20}
    Let \(\mathbb{S}\) be a set containing exactly \(n\) elements. For any non-negative integer \(k\), the number of subsets \(\mathbb{S}\) containing precisely \(k\) elements \(\binom{n}{k}\).
\end{theorem}

\begin{proof}
    Let \(\mathbb{S}\) be a set containing exactly \(n\) elements. For any non-negative integer \(k\), the number of subsets \(\mathbb{S}\) containing precisely \(k\) elements is \(\binom{n}{k}\).

    Suppose that \(\mathbb{S} = \{1, 2, \ldots, n\}\). These numbers may be listed
    in various orders, called permutations, here denoted by \(\pi\). There are
    \(n!\) of these permutations \(\pi\), because the first term may be any one of
    the \(n\) numbers, the second term any one of the \(n - 1\) remaining numbers,
    the third term any one of the still remaining \(n - 2\) numbers, and so on.

    We count the permutations in a way that involves the number \(X\) of subsets
    containing precisely \(k\) elements. Let \(N\) be a specific subset of
    \(\mathbb{S}\) with \(k\) elements. There are \(k!\) permutations of the
    elements of \(N\), each permutation having \(k\) terms. Similarly, there are
    \((n - k)!\) permutations of the \(n - k\) elements not in \(N\). If we attach
    any one of these \((n - k)!\) permutations to the right end of any one of the
    \(k!\) previous permutations, the ordered sequence of \(n\) elements thus
    obtained is one of the permutations \(\pi\) of \(\mathbb{S}\). Thus we can
    generate \(k!(n - k)!\) of the permutations \(\pi\) in this way. To get all the
    permutations \(\pi\) of \(\mathbb{S}\), we repeat this procedure with \(N\)
    replaced by each of the subsets in question. Let \(X\) denote the number of
    these subsets. Then there are \(k!(n - k)!X\) permutations \(\pi\), and
    equating this to \(n!\) we find that
    \[
        X = \frac{n!}{k!(n - k)!}.
    \]

    We now see that the quotient \(\frac{n!}{k! (n - k)!}\) is an integer, because
    it represents the number of ways of doing something. In this way, combinatorial
    interpretations can be useful in number theory.
\end{proof}
\begin{theorem}\label{1.21}
    The product of any \(k\) consecutive integers is divisible by \(k!\).
\end{theorem}
\begin{proof}
    Let's write the product as \(n(n - 1) \cdots (n - k + 1)\). If \(n \geq k\), then we write this in the form \(\binom{n}{k} \cdot k!\) and note that \(\binom{n}{k}\) is an integer, by \cref{1.20}. If \(0 \leq n < k\), then one of the factors of our product is 0, so the product vanishes, and is therefore a multiple of \(k!\) in this case also. Finally, if \(n < 0\), we note that the product may be written as
    \[
        (-1)^k (-n)(-n + 1) \cdots (-n + k - 1) = (-1)^k \binom{-n + k - 1}{k} k!.
    \]
    Note that in this case the upper member \(-n + k - 1\) is at least \(k\), so
    that by \cref{1.20} the binomial coefficient is an integer.

    In the formula for the binomial coefficients we note a symmetry:
    \[
        \binom{n}{k} = \binom{n}{n - k}.
    \]
\end{proof}

\begin{theorem}[The Binomial Theorem]\label{1.22}
    For any integer \(n \geqslant 1\), and any real numbers \(x\) and \(y\), we have
    \[
        (x + y)^n = \sum_{k = 0}^{n} \binom{n}{k} x^k y^{n - k}.
    \]
\end{theorem}
\begin{proof}
    We first consider the product and obtain
    \[\prod_{ i = 1}^{n} (x_i + y_i)\]
    On multiplying this out, we obtain \(2^n\) monomial terms of the form
    \[
        \prod_{i \in \mathbb{A}} x_i \prod_{j \in \mathbb{A}} y_j
    \]
    where \(\mathbb{A}\) is any subset of \{1, 2, \ldots, n\}. For each fixed \(k,
    0 \leqslant k \leqslant n\), we consider the monomial terms obtained from those
    subsets of \(\mathbb{A}\) of \{1, 2, 3, \ldots, n\} having exactly \(k\)
    elements. The number of such subsets is \(\binom{n}{k}\), and the set \(x_i =
    x\) and \(y_i = y\) for all \(i\) and note that such a monomial has a value of
    \(x^k y^{n - k}\) for the subsets in question. Since there are \(\binom{n}{k}\)
\end{proof}
\section{January 24, 2025}
The binomial theorem can also be proved analytically by appealing the following
simple results.
\begin{lemma}\label{1.23}
    Let \(P(z) = \sum_{k = 0}^{n} a_k z^k\) be a polynomial with real coefficients. Then \(a_r = \frac{P^{(r)}(0)}{r!}\) for \(r = 0, 1, 2, \ldots, n\), where \(P^{(r)}(0)\) denotes the \(r^{th}\) derivative of \(P(z)\) evaluated at \(z = 0\).
\end{lemma}

The binomial coefficients arise in many identities, The simplest relations is
the recurrence relation
\[
    \binom{n}{k} + \binom{n}{k + 1} = \binom{n + 1}{k + 1}
\]
Maybe used in many ways, for example to construct the Pascal's Triangle which
is the infinite array of numbers. The pascal's triangle could be used to expand
the binomial theorem, for example \({(x + y)}^5 = 1x^5 + 5x^4y + 10x^3y^2 +
10x^2y^3 + 5xy^4 + 1y^5\) \\
\begin{center}
    \begin{tikzpicture}
        \node at (0, 4) {1};
        \node at (-1, 3) {1};
        \node at (1, 3) {1};
        \node at (-2, 2) {1};
        \node at (0, 2) {2};
        \node at (2, 2) {1};
        \node at (-3, 1) {1};
        \node at (-1, 1) {3};
        \node at (1, 1) {3};
        \node at (3, 1) {1};
        \node at (-4, 0) {1};
        \node at (-2, 0) {4};
        \node at (0, 0) {6};
        \node at (2, 0) {4};
        \node at (4, 0) {1};
        \node at (-5, -1) {1};
        \node at (-3, -1) {5};
        \node at (-1, -1) {10};
        \node at (1, -1) {10};
        \node at (3, -1) {5};
        \node at (5, -1) {1};
    \end{tikzpicture}
\end{center}

This is also obtained by the proceeding row, just to left and just to the
right. In general the \(n^{th}\) row is the coefficients of the expansion of
\((x + y)^{n - 1}\)

\subsection{Congruences}
\subsubsection{Congruences}
A congruence is nothing more than the statement about divisibility.
\begin{definition}
    If an integer \(m \neq 0\), divide the difference \(a - b\), then we say that \underline{\(a\) is congruent to \(b\) modulo \(m\)}, and write we will write \(a \equiv (b \mod{m})\). If \(a - b\) is not divisible by \(n\), we say that \(a\) is not congruent to \(b\) modulo \(m\), and write \(a \not\equiv b \mod{m}\).
\end{definition}

\begin{fact}
    Since \(a - b\) is divisible by \(m\), if \(a - b\) is divisible by \(-m\), we generally take the remainder to be the smallest positive integer.
\end{fact}

\begin{theorem}\label{2.1}
    Let \(a, b, c, d \in \mathbb{Z}\). Then
    \begin{enumerate}
        \item \(a \equiv b \pmod{m}\), \(b \equiv a \pmod{m}\), and \(a - b \equiv 0 \pmod{m}\) are equivalent statements.
        \item If \(a \equiv b \pmod{m}\) and \(b \equiv c \pmod{m}\), then \(a \equiv c
              \pmod{m}\).
        \item If \(a \equiv b \pmod{m}\) and \(c \equiv d \pmod{m}\), then \(a + c \equiv b +
              d \pmod{m}\).
        \item If \(a \equiv b \pmod{m}\) and \(c \equiv d \pmod{m}\), then \(ac \equiv bd
              \pmod{m}\).
        \item If \(a \equiv b \pmod{m}\) and \(d \mid m\), \(d > 0\), then \(a \equiv b
              \pmod{d}\).
        \item If \(a \equiv b \pmod{m}\), then \(ac \equiv bc \pmod{mc}\) for \(c > 0\).
    \end{enumerate}
\end{theorem}

\begin{theorem}\label{2.2}
    Let \(f\) denote a polynomial with integral coefficients. If \(a \equiv b \pmod{m}\), then \(f(a) \equiv f(b) \pmod{m}\).
\end{theorem}
\begin{proof}
    We can suppose \( f(x) = c_n x^n + c_{n-1} x^{n-1} + \cdots + c_0 \) where the \( c_i \) are integers. Since \( a \equiv b \pmod{m} \), we can apply \cref{2.1}, part 4, repeatedly to find \( a^2 \equiv b^2 \), \( a^3 \equiv b^3 \), \(\ldots\), \( a^n \equiv b^n \pmod{m} \), and then \( c_i a^j \equiv c_i b^j \pmod{m} \) and finally \( c_n a^n + c_{n-1} a^{n-1} + \cdots + c_0 \equiv c_n b^n + c_{n-1} b^{n-1} + \cdots + c_0 \pmod{m} \).
\end{proof}

\begin{theorem}\label{2.3}
    For \(a, b, m \in \mathbb{Z}\), \(m > 0\), the situations hold:
    \begin{itemize}
        \item \(ax \equiv ay \pmod{m}\) if an only if \(x \equiv y \pmod{\frac{m}{(a, m)}}\)
        \item If \(ax \equiv ay \pmod{m}\) and \((a, m) = 1\), then \(x \equiv y \pmod{m}\).
        \item \(x \equiv y\pmod{m_i}\) for \(i = 1, 2, \ldots, r\) if and only if \(x \equiv y \pmod{[m_1, m2, \ldots, m]}\)
    \end{itemize}
\end{theorem}

\begin{proof}
    \begin{itemize}
        \item If \(ax \equiv ay \pmod{m}\), then \(ay - ax = mz\) for some integer \(z\).
              Hence we have
              \[
                  a(y - x) = mz,
              \]
              and thus
              \[
                  \frac{a}{(a, m)} (y - x) = \frac{m}{(a, m)} z.
              \]
              But \(\left(\frac{a}{(a, m)}, \frac{m}{(a, m)}\right) = 1\) by \cref{1.7} and
              therefore \(\frac{m}{(a, m)} \mid (y - x)\) by \cref{1.10}. That is,
              \[
                  x \equiv y \pmod{\frac{m}{(a, m)}}.
              \]
        \item Conversely, if \(x \equiv y \pmod{\frac{m}{(a, m)}}\), we multiply by \(a\) to
              get \(ax \equiv ay \pmod{a \cdot \frac{m}{(a, m)}}\) by use of \cref{2.1}, part
              6. But \((a, m)\) is a divisor of \(a\), so we can write \(ax \equiv ay
              \pmod{m}\) by \cref{2.1}, part 5.
    \end{itemize}
    For example, \(15x \equiv 15y \pmod{10}\) is equivalent to \(x \equiv y \pmod{2}\), which amounts to saying that \(x\) and \(y\) have the same parity.
\end{proof}

\begin{proof}
    \begin{itemize}
        \item If \(x \equiv y \pmod{m_i}\) for \(i = 1, 2, \ldots, r\), then \(m_i \mid (y -
              x)\) for \(i = 1, 2, \ldots, r\). That is, \(y - x\) is a common multiple of
              \(m_1, m_2, \ldots, m_r\), and therefore (see \cref{1.12}) \([m_1, m_2, \ldots,
                      m_r] \mid (y - x)\). This implies \(x \equiv y \pmod{[m_1, m_2, \ldots,
                          m_r]}\).
        \item If \(x \equiv y \pmod{[m_1, m_2, \ldots, m_r]}\), then \(x \equiv y
              \pmod{m_i}\) by \cref{2.1} part 5, since \(m_i \mid [m_1, m_2, \ldots, m_r]\).
    \end{itemize}
\end{proof}
\section{January 27, 2025}
In deadling with integers modulo \(m\), we are essentially peforming the
aritmetic but are disregarding the multiples of \(m\). In a sense, not
disregarding between \(a\) and \(a + mx\), where \(x \in \mathbb{Z}\). Given
any integer, \(a\), let \(q\) and \(r\) be the quotient and the remainder on
\(m\); thus \(a = qm + r\). Now \(a \equiv r \pmod{m})\), and since \(r\)
satistifies the inequalities \(0 \leqslant r < m\), we see that every integer
is congruent modulo \(m\) to one of the values \(0, 1, 2, \ldots, m - 1\).
Also, it is clear that no two of these \(m\) integers are congruent modulo
\(m\). These \(m\) values constitute a complete residue system modulo \(m\),
and we now give a general definition of this term.

\begin{definition}
    If \(x \equiv y \pmod {m}\) then \(y\) is called a residue of \(x\) \vocab{modulo} \(m\). A set \(x_1, x_2, \ldots, x_m\) is called a complete residue system modulo \(m\) if for every integer \(y\) there is one and only \(x_j\) such that \(y \equiv x_j\pmod {m}\).
\end{definition}

It is obvious that there are infinitely many complete residue systems
\vocab{modulo} m, the set \(1, 2, \ldots, m - 1, m\) being another example.

A set of \(m\) integers forms a complete residue system modulo \(m\) if and
only if no two integers in the set are congruent modulo \(m\).

For fixed integer \(x \equiv a \pmod {m}\) is the arithmetic progression
\[
    \ldots, a - 3m, a - 2m, a - m, a, a + m, a + 2m, a + 3m, \ldots
\]
This set is called a \vocab{residue class} or \vocab{congruence class} modulo
\(m\). There are \(m\) distinct residue classe modulo \(m\), obtained from
taking \(a = 0, 1, 2, \ldots, m \).
\begin{theorem}\label{2.4}
    If \(b \equiv c \pmod{m}\), then \((b, m) = (c, m)\).
\end{theorem}

\begin{proof}
    We have \(c = b + mx\) for some \(x \in \mathbb{Z}\). Let \(d = (b, m)\). Then \(d \mid b\) and \(d \mid m\). Since \(d \mid m\), we have \(d \mid mx\). Therefore, \(d \mid (b + mx)\), which implies \(d \mid c\). Thus, \(d\) is a common divisor of \(c\) and \(m\), so \(d \leq (c, m)\).

    Conversely, let \(d' = (c, m)\). Then \(d' \mid c\) and \(d' \mid m\). Since
    \(c = b + mx\), we have \(d' \mid (b + mx)\). But \(d' \mid m\), so \(d' \mid
    b\). Thus, \(d'\) is a common divisor of \(b\) and \(m\), so \(d' \leq (b,
    m)\).

    Therefore, \((b, m) = (c, m)\).
\end{proof}

\begin{definition}
    A \vocab{reduced residue system} modulo \(m\) is a set of integers \(r_i\) such that \((r_i, m) = 1, r_i \not\equiv r_j \pmod m\) if \(i \neq j\), and such that every \(x\) prime to \(m\) is congruent modulo \(m\) to some member \(r_i\) of the set.
\end{definition}

\begin{remark}
    In view of the preceding theorem, it is clear that a reduced residue system modulo \(m\) can be obtained by deleting from a complete residue system modulo \(m\) and those members that are not relatively prime to \(m\). Furthermore, all reduced residue system modulo \(m\) have the same number of members, namely \(\phi(m)\). This is called \vocab{Euler's \(\phi\) function}, sometimes called the \vocab{totient function}. By applying the definition of \(\phi(m)\), we can see that \(\phi(p) = p - 1\) for any prime \(p\).
\end{remark}

\begin{theorem}\label{2.5}
    The number \(\phi (m)\) is the number of positive integers less than or equal to \(m\) that are relatively prime to \(m\).
\end{theorem}

Euler's function \(phi (m)\) is of considerable interest. We will consider that
in further sections.
\begin{theorem}\label{2.6}
    Let \(a, m = 1\). Let \(r_1, r_2, \ldots, r_n\) be a complete, or a reduced residue system modulo \(m\). Then \(ar_1, ar_2, \ldots, ar_n\) is a complete, or a reduced, residue system, respectively, modulo \(m\).
\end{theorem}
\begin{proof}
    If \(r_i, m = 1\), then \(ar_i, m\) = 1. There are the same number of \(ar_1, ar_2, \ldots, ar_n\) as of \(r_1, r_2, \ldots, r_n\). Therefore, we need to only show that \(ar_i \not \equiv ar_j \pmod{m}\) if \(i \neq j\). But \cref{2.3} shows that \(ar_i \equiv ar_j \pmod{m}\) implies \(r_i \equiv r_j \pmod{m}\), hence \(i = j\).
\end{proof}
\begin{example}
    For example, since \(1, 2, 3, 4\) is a reduced residue system modulo \(5\), so also is \(2, 4, 6, 8\). Since \(1, 3, 7, 9\) is a reduced residue system modulo \(10\), so also is \(3, 9, 21, 27\).
\end{example}

\begin{theorem}[Fermat's Little Theorem]\label{2.7}
    If \(p \nmid a\), then \((a, p) = 1\) and \(a^{p-1} \equiv 1 \pmod{p}\). To find \(\varphi(p)\), we refer to \cref{2.5}. All the integers \(1, 2, \ldots, p - 1\) are relatively prime to \(p\). Thus we have \(\varphi(p) = p - 1\), and the first part of Fermat's theorem follows. The second part is now obvious.

\end{theorem}

\section{January 29, 2025}
\begin{theorem}[Euler's generalization of Fermat's Theorem]\label{2.8}
    If \((a, m) = 1\) then \[a^{\varphi(m)} \equiv 1\pmod{m}\]
\end{theorem}
\begin{proof}
    Let \(r_1, r_2, \ldots, r_{\varphi(m)}\) be a reduced residue system modulo \(m\). Then by \cref{2.6}, \(ar_1, ar_2, \ldots, ar_{\varphi(m)}\) is also a reduced residue system modulo \(m\). Hence, corresponding to each \(r_i\) there is one and only one \(ar_j\) such that \(r_i \equiv ar_j \pmod{m}\). Furthermore, different \(r_i\) will have different corresponding \(ar_j\). This means that the numbers \(ar_1, ar_2, \ldots, ar_{\varphi(m)}\) are just the residues modulo \(m\) of \(r_1, r_2, \ldots, r_{\varphi(m)}\), but not necessarily in the same order. Multiplying and using \cref{2.1}, part 4, we obtain
    \[
        \prod_{j=1}^{\varphi(m)} ar_j \equiv \prod_{i=1}^{\varphi(m)} r_i \pmod{m},
    \]
    and hence
    \[
        a^{\varphi(m)} \prod_{j=1}^{\varphi(m)} r_j \equiv \prod_{j=1}^{\varphi(m)} r_j \pmod{m}.
    \]
    Now \((r_j, m) = 1\), so we can use \cref{2.3}, part 2, to cancel the \(r_j\)
    and we obtain \(a^{\varphi(m)} \equiv 1 \pmod{m}\).
\end{proof}

\begin{theorem}\label{2.9}
    If \((a, m) = 1\). then there is an \(x\) such that \(ax = 1 \pmod{m}\) and any two such \(x\) are congurent \(pmod{m}\). If \((a, m) > 1\), then there is no such \(x\).
\end{theorem}

\begin{proof}
    If \(a, m = 1\), then there exist \(x\) and \(y\) such that \(ax + my = 1\) That is. \(ax \equiv 1\), Conversely, if \(ax \equiv 1 \pmod{m}\), then there is a \(y\) such that \(ax + by = 1\), so that \((a, m) = 1\). Thus if, \(ax_1 \equiv ax_2 \equiv 1 \pmod{m}\), then \((a, m) = 1\), and that follows from \cref{2.3}, part 2.
\end{proof}
\begin{note}

    The relation \(ax \equiv 1 \pmod{m}\) asserts that there is a residue system x
    that is multiplicative inverse of the class \(a\). To avoid confusion rational
    number \(a^{-1} = \frac{1}{m}\), we denote that this residue \(\bar{a}\). The
    value of \(\bar{a}\) is quickly found by employing the Eucledian Algorithm, as
    asserted. The existence of \(\bar{a}\) is also evident from \cref{2.6}, if
    \((a, m) = 1\) then the members \(a, 2a, \ldots, ma\) form a complete system of
    residues, which is to say, that is one of them is \(\equiv 1 \pmod{m}\). In
    additional it can be inferred in the form \(\bar{a} = a^{\varphi(m)} - 1\)
\end{note}\

\begin{lemma}\label{2.10}
    Let \(p\) be a prime number. Then \(x^2 \equiv 1 \pmod{m} \Longleftrightarrow x = \pm 1 \pmod{m}\). In a later section, we will establish a more general result which the following is easily derived, but we are giving a direct proof for now, because the observation has many useful applications.
\end{lemma}
\begin{proof}
    This is a quadratic congruence. It may be expressed as \(x^2  - 1\equiv 0 \pmod{m}\). That is \((x - 1)(x - 2) \equiv 0 \pmod{p}\), which is to say that \(\mid (x - 1)(x - 1)\mid\). By \cref{1.15} it follows that \(p \mid (x - 1)\) or \(p \mid (x + 1)\). So \(x \equiv 1 \pmod{m}\) or \(x \equiv -1 \pmod{m}\) .Conversely, it either
\end{proof}

\begin{theorem}[Wilson's Theorem]\label{2.11}
    If \(p\) is a prime, then \((p  - 1) \equiv -1 \pmod{m}\)
\end{theorem}

\section{January 31, 2025}
\begin{proof}
    If \(p = 2\) or \(p = 3\), the congruence is easily verified. Thus we may assume that \(p \geq 5\). Suppose that \(1 \leq a \leq p - 1\). Then \((a, p) = 1\), so that by~\cref{2.9} there is a unique integer \(\bar{a}\) such that \(1 \leq \bar{a} \leq p - 1\) and \(a\bar{a} \equiv 1 \pmod{p}\). By a second application of \cref{2.9} we find that if \(a\) is given then there is exactly one \(\bar{a}\), \(1 \leq \bar{a} \leq p - 1\), such that \(a\bar{a} \equiv 1 \pmod{p}\). Thus \(a\) and \(\bar{a}\) form a pair whose combined contribution to \((p - 1)!\) is \(\equiv 1 \pmod{p}\). However, a little care is called for because it may happen that \(a = \bar{a}\). This is equivalent to the assertion that \(a^2 \equiv 1 \pmod{p}\), and by \cref{2.10} we see that this is in turn equivalent to \(a \equiv 1\) or \(a \equiv p - 1\). That is, \(\bar{1} = 1\) and \(\overline{p - 1} = p - 1\), but if \(2 \leq a \leq p - 2\) then \(a \neq \bar{a}\). By pairing these latter residues in this manner we find that \(\prod_{a=2}^{p-2} a \equiv 1 \pmod{p}\), so that \((p - 1)! \equiv 1 \cdot \prod_{a=2}^{p-2} a \cdot (p - 1) \equiv -1 \pmod{p}\).
\end{proof}
\begin{theorem}\label{2.12}
    Let \(p\) denote a prime. Then \(x^2 \equiv -1 \pmod {m}\) has solution \(\Longleftrightarrow p = 2\) or \(p \equiv 1 \pmod{4}\)
\end{theorem}

\begin{proof}
    If \(p = 2\), we have the solution \(x = 1\). FOr any odd prime \(p\), we can write Wilson's theorem in the form

    \[\left(1 \cdot 2 \ldots j \ldots \frac{p - 1}{2}\right)\left(\frac{p + 1}{2} \ldots (p - j) \ldots (p - 2)(p - 1)\right) \equiv -1 \pmod{p}\]
    The product on the left has divided into two parts, each with the same number
    of factors. Pairing off \(j\) in the first half with \(p - j\) in the second
    half, we can rewrite the congruence in the form
    \[\prod_{j = 1}^{\frac{p - 1}{2}} j (p - j)\equiv \pmod{p}\]
    But \(j(p - j) \equiv -j^2 \pmod{p}\), and so the above is
    \[\prod_{j = 1}^{\frac{p - 1}{2}}(-j^2) \equiv (-1)^{\frac{p - 1}{2}}\left(\prod_{j = 1}^{\frac{p - 1}{2}} j \pmod{p}.\right)\]
    If \(p \equiv 1 \pmod{4}\) then the first factor on the right is 1, and we see
    that \(x = \left(\frac{p - 1}{2}\right)!\) is a solution of \(x^2 \equiv -1
    \pmod{p}\).

    Suppose, conversely, that there is an \(x\) such that \(x^2 \equiv -1
    \pmod{p}\). We note that for such an \(x, p \nmid x\). We suppose that \(p >
    2\), and raise both sides of the congruence to the power \(\frac{p - 1}{2}\) to
    see that
    \[(-1)^\frac{p - 1}{2} \equiv (x^2)^{\frac{ p - 1}{2}} = x^{p - 1} \pmod{p}\].

    By Fermat's congruence, the right side here is \(\equiv 1 \pmod{p}\). The left
    hand side is \(\pm 1\). Since \(-1 \not \equiv 1 \pmod{m}\), we deduce that \[(-1)^{\frac{p - 1}{2}} = 1.\] Thus \(\frac{p - 1}{2}\) is even; that is, \(p \equiv 1 \pmod{4}\).

    In the case \(p \equiv 1 \pmod {4}\), we have expilcitly constructed a solution
    of the congruence, \(x^2 \equiv -1 \pmod{p}\). However, the amout of
    calculation required to evaluate \(\frac{p - 1}{2}! \pmod{p}\) is no smaller
    than the exhausting \(x = 1, x 2, \ldots, x = \frac{p - 1}{2}\). In a later
    section, we will develop a method by which the desired \(x\) can be quickly
    determined.
\end{proof}

\section{February 3, 2025}

\cref{2.12} provides a key piece of information needed to determine
which integers can be written as the sum of two squares. We began by showing
that a a class of prime numbers can be represented in this manner.

\begin{lemma}\label{2.13}
    If \(p\) is a prime numebr and \(p \equiv 1 \pmod{4}\) then there exist positive integers \(a\) and \(b\) such that \(a^2 + b^2 = p.\) This was first stated in 1632 by Albert Girard on the basis of numericla evidence. The first proof was given by Fermat in 1654.
\end{lemma}

\begin{lemma}\label{2.14}
    Let \(q\) be prime of the form \(a^2 + b^2\). If \(q \equiv 3 \pmod{4}\). then \(q \mid a\) and \(q \mid b\).
\end{lemma}

\begin{theorem}[Fermat]\label{2.15}
    Write the canonical factorization of \(n\) in the form

    \[n = 2^{\alpha} \prod_{p \equiv 1   (4)} p ^{\beta}\prod_{p \equiv 3   (4)} q ^{\gamma}\]
    Then \(n\) can be expressed as a sum of the two squres \(\Longleftrightarrow\)
    all exponents \(\gamma\) are there.
\end{theorem}

\begin{note}
    We note that the identity holds:
    \[(a^2 + b^2 ) (c^2 + d^2 ) = (ac - bd)^2 + (ac + bc)^2\]
\end{note}
The Theorem of Fermat is the first of many such theorems. The object of constructing a coherent theorey of quadratic forms was the primary in the instance on research for seveveral centuries. This first setep in the theory is to generate \cref{2.12}. This is accomplished in the law of quadratic reciprocity, whcih we study in the initial chapters of the following chapters. With this tool in hand, we deelop some of the few fundamentals concerting quadratic forms in the latter part of Chapter 3. In particular, in sections, we apply the general theory of the sum of two squares, to give not only a proof of \cref{2.15} but also some further results.

\subsection[Solutions of Congruences]{Solutions of Congruences}
Let \(f(x)\) denote a polynomial with the integer coefficients
\[f(x) = a_n x^n + a_{n + 1}x^{n - 1} + \cdots + a_0.\] If \(n\) is an integer such that \(f(u) \equiv 0 \pmod{m}\) we say that it is a
solution of the congruence \(f(x) \equiv 0 \pmod{m}\). Whether or not an
integer \(a\) is a solution of a congruence depends on the modulo \(m\).

If the integer \(u\) is a solution of \(f(x) \equiv 0 \pmod{m}\) and if \(v
\equiv u \pmod{m}\), then \cref{2.2} shows that \(v\) is also a solution.
Because of this we shall say that \(f(x) \equiv 0 \pmod{m}\) meaning that every
integer congruent to \(u \pmod{m}\) satisfies \(f(x) \equiv 0 \pmod{m}\).
\begin{example}
    The congruence \(x^2 - x + 4 \equiv 0 \pmod{10}\) has the solution \(x = 3\) and the solution \(x = 8\). It also has solutions \(x = 13\) and \(x = 18\) and all other numbers obtained by adding and subtracting 10 as often as we wish. In counting the number of solutions of a congruence, we can restrict our attention to complete residue system belong to the modolus. In the example \(x^2 - x + 4 \equiv 0 \pmod{10}\) because \(x = 3\) and \(x = 8\) are the only numbers among \(0, 1, 2, \ldots, 9\) that are solutions. The two solutions can be written in the form \(x = 3\) or \(x = 8\) on in congruence from \(x \equiv 3 \pmod{10}\) and \(x \equiv 8 \pmod{10}\).
\end{example}

\begin{example}
    The congruence
    \[x^2 - 7x + 2 \equiv 0 \pmod{10}\]
    has exactly 4 solutions, \(x = 3, 4, 8, 9\). The reason for counting the number
    of solutions in this way is that if \(f(x) \equiv 0 \pmod{m}\) has a solution
    \(x = a\), then it follows that all integers \(x\) satisfying \(x \equiv a
    \pmod{m}\) are automatically solutions, so this entire congruenec class is
    counted as a single solution.
\end{example}

\begin{definition}
    Let \(r_1, r_2, \ldots, r_m\) denote a complete system of residues \(\pmod{m}\) Then the number of solutions of \(f(x) \equiv 0 \pmod{m}\) is the number of solutions \(r_i\) such that \(f(r_1) \equiv 0 \pmod{m}\).
\end{definition}

\section{February 5, 2025}
\begin{example}

    \(x^2 + 1 = 0 \pmod{7}\) has no solutions. \\
    \(x^2 + 1 = 0 \pmod{5}\) has two solutions. \\
    \(x^2 - 1 = 0 \pmod{8}\) has \(4\) solutions.

\end{example}

\begin{definition}\label{Definition 2.5}
    Let \[f(x) = a_{n} x^n + a_{n-1}x^{n - 1} + \cdots + a_0\]
    If \(a_n \not \equiv 0 \pmod{m}\), then the degree of the congruence \(f(x) = 0
    \pmod{m}\) is degree \(n\). \\ If \(a_n \equiv 0 \pmod{m}\). Then let \(j\) be
    the largest integer such that \(a_j \not \equiv 0 \pmod{m}\). \\ If there is no
    such \(j\), so all coefficients are multiples of \(m\), then the degree is not
    defined to the congruence. It should be noted that the degree of the congruence
    \(f(x) \equiv 0 \pmod{m}\) is not the same as the degree fo the polynomial
    \(f(x)\). \\ The degree of the congruence depends on the modulus \(m\) and the
    coefficients of the polynomial \(f(x)\).
\end{definition}

\begin{example}
    If \[g(x)=6x^3 + 3x^2 + 1\] then \(g(x) \equiv 0 \pmod{5}\) has degree three but \(g(x) \equiv 0 \pmod{2}\)
    has degree \(2\) where as \(g(x)\)is of degree \(3\).
\end{example}

\begin{theorem}\label{2.16}
    If \(d \mid m, d > 0\), and if the solution of \(f(x) \equiv 0 \pmod{m}\) then it is a solution of \(f(x) \equiv 0 \pmod{d}\).
\end{theorem}

\begin{proof}
    This follows directly from \cref{2.1}, part 5
\end{proof}
This is a distinction mode in the theorey of algebraic congruence equations that has an analogy for congruences. A conditional equation such as \(x^2 - 5x + 6 = 0\) is true only for certain values of \(x\), namely \(x = 2\) and \(x = 3\). An identity of identical equations, such as \((x - 2)^2 = x^2 - 4x + 4\) holds for all real numbers of complex numbers. Similarly, we say \(f(x) \equiv 0 \pmod{m}\) is an \vocab{identical congruence} if all polynomials all of those coefficients are divisble by all whose coefficients are divisible by \(f(x) \equiv 0 \pmod{m}\) is an identical congruence. A different type of identical congruence is also illustrated by \(x^p \equiv x \pmod{p}\) which is trye by Fermat's theorem.\\
So before, considering congruences of higher degree, we first descibe the solutions in the linear case.
\begin{theorem}\label{2.17}
    Let \(a^b\), and \(m > 0\) be integers. Put \(g = (a, m)\)and now the congruence \(ax \equiv b \pmod{m}\) has a solution \(\Leftrightarrow g \mid b\). If the condition is met, then the solution from an arithmetic property progressoin with common differnece \(m / g\), giving the solutions \(\pmod{m}\).
\end{theorem}
\begin{proof}
    The question is whether there exist integers \(x\) and \(y\) such that \(ax + my = b\). Since \(g\) divides the left side, for such integers to exist we must have \(g \mid b\). Suppose that this condition is met, and write \(a = g\alpha\), \(b = g\beta\), \(m = g\gamma\). Then by the first part of \cref{2.3}, the desired congruence holds if and only if \(\alpha x \equiv \beta \pmod{\gamma}\). Here \((\alpha, \gamma) = 1\) by \cref{1.7}, so by \cref{2.9} there is a unique number \(\bar{\alpha} \pmod{\gamma}\) such that \(\alpha \bar{\alpha} \equiv 1 \pmod{\gamma}\). On multiplying through by \(\bar{\alpha}\), we find that \(x = \bar{\alpha} \beta \pmod{\gamma}\). Thus the set of integers \(x\) for which \(ax \equiv b \pmod{m}\) is precisely the arithmetic progression of numbers of the form \(\bar{\alpha} \beta + k\gamma\). If we allow \(k\) to take on the values \(0, 1, \ldots, g - 1\), we obtain \(g\) values of \(x\) that are distinct \(\pmod{m}\). All other values of \(x\) are congruent \(\pmod{m}\) to one of these, so we have precisely \(g\) solutions.
\end{proof}
\begin{fact}
    Since \(\bar \alpha\) can be found by application of the Euclidean algorithm, we have a method for finding all solutions of \(ax \equiv b \pmod{m}\) when \(g \mid b\).
\end{fact}

\subsection{Chinese Remainder Theorem}
We now consider the important problem of solving simultaneous congruences. The
simplest case of this is to see if there is any \(x\) that satisfies the
simultaneous congruences:

\[
    \begin{aligned}\label{congruence}
        x & \equiv a_1 \pmod{m_1} \\
        x & \equiv a_2 \pmod{m_2} \\
          & \vdots                \\
        x & \equiv a_r \pmod{m_r}
    \end{aligned}
\]
This is the subject of the Chinese Remainder Theorem because it was known in
China in the first century AD.

\begin{theorem}[Chinese Remainder Theorem]\label{2.18}
    Let \(m_1, m_2, \ldots, m_r\) denote \(r\) positive integers that are relatively prime in pairs. Let \(a_1, a_2, \ldots, a_r\) denote any \(r\) integers. If the congruence \ref{congruence} holds that means that \(x\) is in the form of \(x = x_0 + km\) for some integer \(k\). Here, \(m = m_1 m_2 \ldots m_r\).
\end{theorem}

%\section{February 7, 2025}
%\section{February 10, 2025}
%\section{February 12, 2025}
%\section{February 14, 2025}
%\section{February 17, 2025}
%\section{February 19, 2025}
%\section{February 21, 2025}
%\section{February 24, 2025}
%\section{February 26, 2025}
%\section{February 28, 2025}
%\section{March 3, 2025}
%\section{March 5, 2025}
%\section{March 7, 2025}
%\section{March 17, 2025}
%\section{March 19, 2025}
%\section{March 21, 2025}
%\section{March 24, 2025}
%\section{March 26, 2025}
%\section{March 28, 2025}
%\section{March 31, 2025}
%\section{April 2, 2025}
%\section{April 4, 2025}
%\section{April 7, 2025}
%\section{April 9, 2025}
%\section{April 11, 2025}
%\section{April 14, 2025}
%\section{April 16, 2025}
%\section{April 18, 2025}
%\section{April 21, 2025}
%\section{April 23, 2025}
%\section{April 25, 2025}
%\section{April 28, 2025}

\end{document}